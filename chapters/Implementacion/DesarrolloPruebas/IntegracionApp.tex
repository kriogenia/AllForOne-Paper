\subsection{Aplicación móvil}

\begin{longtable}{ l l c }
    \hline
    Referencia & Caso & Resultado \\
    \hline
    \ref{cp:i:app:registro_paciente} & Registro de un Paciente en la aplicación & NO \\
    \ref{cp:i:app:registro_cuidador} & Registro de un Cuidador en la aplicación & NO \\ \hline
    \ref{cp:i:app:inicio_sesion} & Inicio de sesión en la aplicación & NO \\ \hline
    \ref{cp:i:app:mostrar_codigo_vinculacion} & Mostrar código de vinculación en la aplicación & NO \\
    \ref{cp:i:app:desplegar_escaner} & Desplegar escáner en la aplicación & NO \\
    \ref{cp:i:app:eliminiacion_vinculo_paciente} & Eliminación de vínculo de un Paciente & NO \\
    \ref{cp:i:app:eliminacion_vinculo_cuidador} & Eliminación de vínculo de un Cuidador & NO \\ \hline
    \ref{cp:i:app:compartir_ubicacion} & Compartir ubicación en la aplicación & NO \\ \hline
    \ref{cp:i:app:gestionar_tareas_tasks} & Gestionar las tareas desde Tareas & NO \\
    \ref{cp:i:app:gestionar_tareas_feed} & Gestionar las tareas desde el Feed & NO \\ \hline
    \ref{cp:i:app:enviar_recibir_mensajes} & Enviar y recibir mensajes en la aplicación & NO \\ \hline
    \ref{cp:i:app:gestionar_notificaciones} & Gestionar notificaciones en la aplicación & NO \\ \hline
    \ref{cp:i:app:consultar_vinculos} & Consultar los vínculos en la aplicación & NO \\ \hline
    \ref{cp:i:app:consultar_paciente} & Consultar paciente en la aplicación & NO \\ \hline
    \caption{Resultados de las pruebas de integración de la aplicación}
\end{longtable}

No se han podido llevar a cabo pruebas de integración en la aplicación. El equipo de desarrollo no ha sido capaz de lanzar la aplicación con Espresso y conseguir realizar el inicio de sesión a través de la autenticación de Google por problemas con las cuentas de usuario. Ha sido el mayor error del proyecto.