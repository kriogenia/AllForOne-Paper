\chapter{Desarrollo de pruebas}
\label{ch:desarrollo_pruebas}

A continuación se ofrece el resultado de las pruebas propuestas en el \fref{ch:especificacion_tecnica_plan_pruebas}. La tabla lista la referencia del caso de prueba en dicha sección, el título del caso de prueba y el resultado de los mismos en el momento de entrega de este documento. Este resultado puede ser \textbf{PASA}, si la prueba ha sido implementada y se completa con éxito; \textbf{FALLA}, si la prueba ha sido implementada pero no ofrece el resultado esperado; y \textbf{NO}, para las pruebas que no han llegado a ser implementadas en el sistema.

\section{Pruebas unitarias}

\subsection{API}

\begin{longtable}{ l l c }
    \hline
    Referencia & Caso & Resultado \\
    \hline
    \ref{cp:u:api:encabezado_correcto} & Encabezado de autenticación correcto & PASA \\
    \ref{cp:u:api:encabezado_faltante} & Encabezado de autenticación faltante & PASA \\
    \ref{cp:u:api:encabezado_mal_formateado} & Encabezado de autenticación mal formateado & PASA \\
    \ref{cp:u:api:encabezado_token_invalido} & Encabezado de autenticación con token inválido & PASA \\\hline
    \ref{cp:u:api:manejo_error_correcto} & Manejo correcto de errores conocidos & PASA \\
    \ref{cp:u:api:manejo_correcto_desconocido} & Manejo correcto de errores desconocidos & PASA \\\hline
    \ref{cp:u:api:crear_mensaje_texto} & Creación de un mensaje de texto & PASA \\
    \ref{cp:u:api:crear_mensaje_texto_invalido} & Creación de un mensaje de texto inválido & PASA \\
    \ref{cp:u:api:recuperar_mensaje} & Recuperación de mensajes & PASA \\
    \ref{cp:u:api:recuperar_mensaje_pagina} & Recuperación de mensajes de una página concreta & PASA \\\hline
    \ref{cp:u:api:crear_notificacion} & Creación de una notificación & PASA \\
    \ref{cp:u:api:marcar_notificacion_leida} & Marcado de una notificación con más interesados como leída & PASA \\
    \ref{cp:u:api:marcar_notificacion_compartida_leida} & Marcado de una notificación sin más interesados como leída & PASA \\
    \ref{cp:u:api:marcar_notificaciones_leidas} & Marcado de todas las notificaciones como leídas & PASA \\
    \ref{cp:u:api:recuperar_notificaciones} & Recuperación de notificaciones & PASA \\
    \ref{cp:u:api:recuperar_notificaciones_edad} & Recuperación de notificaciones con edad especificada & PASA \\\hline
    \caption{Primera parte de los resultados de las pruebas unitarias de la API}
\end{longtable}

\begin{figure}[H]
\begin{longtable}{ l l c }
    \hline
    Referencia & Caso & Resultado \\
    \hline
    \ref{cp:u:api:inicio_sesion} & Inicio de sesión & PASA \\
    \ref{cp:u:api:cierre_sesion} & Cierre de sesión & PASA \\
    \ref{cp:u:api:comprobar_sesion_abierta} & Comprobación de sesión abierta & PASA \\
    \ref{cp:u:api:comprobar_sesion_cerrada} & Comprobación de sesión cerrada & PASA \\
    \ref{cp:u:api:comprobar_tupla_existente} & Comprobación de tuplas de sesión existentes & PASA \\
    \ref{cp:u:api:comprobar_tupla_inexistente} & Comprobación de tuplas de sesión inexistentes & PASA \\
    \ref{cp:u:api:comprobar_tupla_invalida} & Comprobación de tuplas de sesión inválidas & PASA \\\hline
    \ref{cp:u:api:crear_tarea} & Creación de una tarea & PASA \\
    \ref{cp:u:api:crear_tarea_invalida} & Creación de una tarea inválida & PASA \\
    \ref{cp:u:api:eliminar_tarea} & Eliminación de una tarea & PASA \\
    \ref{cp:u:api:actualizar_tarea} & Actualización de una tarea & PASA \\
    \ref{cp:u:api:actualizar_tarea_no_existente} & Actualización de una tarea no existente & PASA \\
    \ref{cp:u:api:recuperar_tareas} & Recuperación de tareas relevantes & PASA \\
    \ref{cp:u:api:recuperar_tareas_edad} & Recuperación de tareas relevantes de edad especificada & PASA \\
    \ref{cp:u:api:comprobar_sala_tarea} & Comprobación de la sala de una tarea & PASA \\\hline 
    \ref{cp:u:api:generar_token_sesion} & Generación de tokens de sesión & PASA \\
    \ref{cp:u:api:refresco_token_sesion} & Refresco de tokens de sesión & PASA \\
    \ref{cp:u:api:refresco_token_sesion_invalido} & Refresco de tokens de sesión inválida & PASA \\
    \ref{cp:u:api:generacion_token_vinculacion} & Generación de tokens de vinculación & PASA \\
    \ref{cp:u:api:decodificacion_token_vinculacion} & Decodificación de tokens de vinculación & PASA \\
    \ref{cp:u:api:decodificacion_token_vinculacion_invalido} & Decodificación de tokens de vinculación inválidos & PASA \\
    \ref{cp:u:api:comprobar_tupla_token} & Comprobación de tuplas de tokens & PASA \\\hline
    \ref{cp:u:api:recuperar_usuario_no_existente_googleid} & Recuperación de un usuario no existente por su ID de Google & PASA \\
    \ref{cp:u:api:recuperar_usuario_googleid} & Recuperación de un usuario existente por su ID de Google & PASA \\
    \ref{cp:u:api:actualizar_usuario} & Actualización de un usuario & PASA \\
    \ref{cp:u:api:crear_vinculo} & Creación de vínculo & PASA \\
    \ref{cp:u:api:crear_vinculo_invalido} & Creación de vínculo inválida & PASA \\
    \ref{cp:u:api:eliminar_vinculo} & Eliminación de vínculos inválida & PASA \\
    \ref{cp:u:api:recuperar_cared} & Recuperación de Paciente vinculado de un Cuidador & PASA \\
    \ref{cp:u:api:recuperar_cared_no_vinculo} & Recuperación de Paciente vinculado de un Cuidador no vinculado & PASA \\
    \ref{cp:u:api:recuperar_cared_invalido} & Recuperación de Paciente vinculado inválida & PASA \\
    \ref{cp:u:api:recuperar_vinculos} & Recuperación de vínculos & PASA \\
    \ref{cp:u:api:recuperar_rol} & Recuperación de rol de un usuario & PASA \\\hline
    \caption{Segunda parte de los resultados de las pruebas unitarias de la API}
\end{longtable}
\end{figure}

Como se puede comprobar las pruebas unitarias de la \acrshort{api} no han podido ofrecer un resultado más satisfactorio. Todos los casos de prueba han sido implementados y sus resultados son exitosos. Además, se ha conseguido una cobertura de código superior al \textbf{90\%} en líneas, condiciones, métodos y clases.
\subsection{Aplicación móvil}

\begin{longtable}{ l l c }
    \hline
    Referencia & Caso & Resultado \\
    \hline
    \ref{cp:u:app:actualizar_datos} & Actualizar datos del usuario & NO \\
    \ref{cp:u:app:cerrar_sesion_usuario} & Cerrar la sesión de usuario & NO \\
    \ref{cp:u:app:eliminar_vinculo} & Eliminar el vínculo con el Paciente vinculado & NO \\ \hline
    \ref{cp:u:app:añadir_marcador} & Añadir un marcador & PASA \\
    \ref{cp:u:app:comprobar_marcador} & Comprobar la existencia de un marcador & PASA \\
    \ref{cp:u:app:eliminar_marcador} & Eliminar un marcador & PASA \\
    \ref{cp:u:app:actualizar_marcador} & Actualizar un marcador & PASA \\ \hline
    \ref{cp:u:app:añadir_notificacion} & Añadir una notificación & NO \\
    \ref{cp:u:app:añadir_conjunto_notificaciones} & Añadir un conjunto de notificaciones & NO \\
    \ref{cp:u:app:limpiar_lista_notificaciones} & Limpiar lista de notificaciones & NO \\
    \ref{cp:u:app:cargar_lista_notificaciones} & Cargar lista de notificaciones & NO \\
    \ref{cp:u:app:eliminar_notificacion} & Eliminar una notificación & NO \\
    \ref{cp:u:app:eliminar_todas_notificaciones} & Eliminar todas las notificaciones & NO \\ \hline
    \ref{cp:u:app:enviar_confirmacion_datos_usuario} & Enviar confirmación de datos de usuario & PASA \\
    \ref{cp:u:app:gestionar_resultado_inicio_google} & Gestionar resultado de inicio de sesión en Google & PASA \\
    \ref{cp:u:app:gestionar_resultado_inicio_google_erroneo} & Gestionar resultado de inicio de sesión en Google erróneo & PASA \\ \hline
    \ref{cp:u:app:añadir_nuevo_marcador_vista} & Añadir un nuevo marcador & PASA \\
    \ref{cp:u:app:recibir_actualizacion_posicion_no_existente} & Recibir una actualización de posición de marcador no existente & NO \\
    \ref{cp:u:app:recibir_actualizacion_posicion_marcador} & Recibir una actualización de posición de marcador ya existente & NO \\ \hline  
    \ref{cp:u:app:obtener_pagina_mensajes} & Obtener una página de mensajes & PASA \\
    \ref{cp:u:app:enviar_tarea_feed} & Enviar una tarea por el feed & NO \\
    \ref{cp:u:app:enviar_mensaje_feed} & Enviar un mensaje por el feed & NO \\
    \ref{cp:u:app:recibir_actualizacion_tarea} & Recibir una actualización de tarea & NO \\
    \ref{cp:u:app:recibir_eliminacion_tarea} & Recibir una eliminación de tarea & NO \\
    \ref{cp:u:app:recibir_nuevo_mensaje} & Recibir un nuevo mensaje & NO \\ \hline
    \caption{Primera parte de los resultados de las pruebas unitarias de la aplicación}
\end{longtable}   

\begin{figure}[H]
\begin{longtable}{ l l c }
    \hline
    Referencia & Caso & Resultado \\
    \hline    
    \ref{cp:u:app:confirmar_creacion_tarea} & Confirmar creación de una tarea & PASA \\
    \ref{cp:u:app:eliminacion_tarea_vista} & Eliminación una tarea & PASA \\
    \ref{cp:u:app:eliminacion_tarea_invalida} & Eliminación una tarea inválida & PASA \\
    \ref{cp:u:app:marcar_tarea_hecha_no_hecha} & Marcar tarea como hecha/no hecha & PASA \\
    \ref{cp:u:app:obtener_lista_tareas} & Obtener la lista de tareas & PASA \\ \hline
    \ref{cp:u:app:eliminar_vinculo_vista} & Eliminar un vínculo & NO \\
    \ref{cp:u:app:obtener_codigo_qr_vinculacion} & Obtener un código QR de vinculación & NO \\
    \ref{cp:u:app:obtener_vinulos_vista} & Obtener los vínculos & PASA \\ \hline
    \ref{cp:u:app:actualizar_usuario_vista} & Actualizar usuario & NO \\
    \ref{cp:u:app:enviar_codigo_vinculacion_vista} & Enviar código de vinculación & PASA \\
    \ref{cp:u:app:obtener_lista_notificaciones_vista} & Obtener lista de notificaciones & NO \\
    \ref{cp:u:app:obtener_paciente_vinculado_vista} & Obtener Paciente vinculado & PASA \\ \hline
    \ref{cp:u:app:envio_mensajes_repo} & Envío de mensaje & NO \\
    \ref{cp:u:app:recuperacion_mensajes_repo} & Recuperación de mensajes & PASA \\ \hline
    \ref{cp:u:app:recuperacion_notificaciones_repo} & Recuperación de notificaciones & PASA \\
    \ref{cp:u:app:marcado_notificacion_leida_repo} & Marcado de notificación como leída & PASA \\
    \ref{cp:u:app:marcado_todas_notificaciones_leidas_repo} & Marcado de todas las notificaciones como leídas & PASA \\ \hline
    \ref{cp:u:app:inicio_sesion_repo} & Inicio de sesión & PASA \\
    \ref{cp:u:app:cierre_sesion_repo} & Cierre de sesión & PASA \\
    \ref{cp:u:app:refresco_sesion_repo} & Refresco de sesión & PASA \\
    \ref{cp:u:app:recuperacion_tareas_repo} & Recuperación de tareas & PASA \\
    \ref{cp:u:app:guardado_tarea_repo} & Guardado de una tarea & PASA \\
    \ref{cp:u:app:eliminacion_tarea_repo} & Eliminación de una tarea & PASA \\
    \ref{cp:u:app:actualizacion_tarea_repo} & Actualización de una tarea & PASA \\ \hline
    \ref{cp:u:app:actualizacion_usuario_repo} & Actualización de un usuario & PASA \\
    \ref{cp:u:app:eliminar_vinculacion_repo} & Eliminar vinculación & PASA \\
    \ref{cp:u:app:enviar_codigo_vinculacion_repo} & Enviar código de vinculación & PASA \\
    \ref{cp:u:app:recuperacion_vinculos_repo} & Recuperación de vínculos & PASA \\
    \ref{cp:u:app:recuperacion_paciente_repo} & Recuperación del Paciente vinculado & PASA \\
    \ref{cp:u:app:solicitar_codigo_vinculacion_repo} & Solicitar código de vinculación & PASA \\ \hline
    \ref{cp:u:app:obtencion_servicio} & Obtención de un servicio & PASA \\ \hline
    \ref{cp:u:app:respuesta_inicio_sesion} & Procesar respuesta de inicio de sesión & PASA \\
    \ref{cp:u:app:respuesta_refresco_sesion} & Procesar respuesta de refresco de sesión & PASA \\ \hline
    \ref{cp:u:app:respuesta_recuperacion_notificaciones} & Procesar respuesta de recuperación de notificaciones & PASA \\
    \ref{cp:u:app:respuesta_recuperacion_mensajes} & Procesar respuesta de recuperación de mensajes & PASA \\ \hline
    \caption{Segunda parte de los resultados de las pruebas unitarias de la aplicación}
\end{longtable}   
\end{figure}    

\begin{figure}[H]
\begin{longtable}{ l l c }
    \hline
    Referencia & Caso & Resultado \\
    \hline  
    \ref{cp:u:app:respuesta_recuperacion_tareas} & Procesar respuesta de recuperación de tareas & PASA \\
    \ref{cp:u:app:respuesta_publicacion_tareas} & Procesar respuesta de publicación de tareas & PASA \\ \hline
    \ref{cp:u:app:respuesta_actualizacion_usuario} & Procesar respuesta de actualización de usuario & PASA \\
    \ref{cp:u:app:respuesta_recuperacion_paciente_vinculado} & Procesar respuesta de recuperación de Paciente vinculado & PASA \\
    \ref{cp:u:app:respuesta_recuperacion_vinculos} & Procesar respuesta de recuperación de vínculos & PASA \\
    \ref{cp:u:app:respuesta_recuperacion_codigo_vinculacion} & Procesar respuesta de recuperación de código de vinculación & PASA \\ \hline
    \caption{Tercera parte de los resultados de las pruebas unitarias de la aplicación}
\end{longtable}  
\end{figure} 

En el desarrollo de pruebas de la aplicación no fueron posibles realizar todas las pruebas que estaban planeadas por limitaciones de tiempo. Dentro de las que sí pudieron implementadas se consiguió un éxito completo. Estas pruebas necesitan mayor trabajo.

\section{Pruebas de integración}

\subsection{API}

\begin{longtable}{ l l c }
    \hline
    Referencia & Caso & Resultado \\
    \hline
    \ref{cp:i:api:registro_usuario_correcto} & Registro de usuario correcto & PASA \\
    \ref{cp:i:api:inicio_sesion_correcto} & Inicio de sesión correcto & PASA \\
    \ref{cp:i:api:inicio_sesion_incorrecto} & Inicio de sesión incorrecto & PASA \\
    \ref{cp:i:api:cierre_sesion_correcto} & Cierre de sesión correcto & PASA \\
    \ref{cp:i:api:cierre_sesion_incorrecto} & Cierre de sesión incorrecto & PASA \\
    \ref{cp:i:api:refresco_sesion_correcto} & Refresco de sesión correcto & PASA \\
    \ref{cp:i:api:refresco_sesion_incorrecto} & Refresco de sesión incorrecto & PASA \\ \hline
    \ref{cp:i:api:conexion_sala_localizacion} & Conexión a la sala de localización & PASA \\
    \ref{cp:i:api:desconexion_sala_localizacion} & Desconexión de la sala de localización & PASA \\
    \ref{cp:i:api:actualizacion_ubicacion} & Actualización de ubicación & PASA \\ \hline
    \caption{Primera parte de los resultados de las pruebas de integración de la API}
\end{longtable}

\begin{figure}[H]
\begin{longtable}{ l l c }
    \hline
    Referencia & Caso & Resultado \\
    \hline
    \ref{cp:i:api:recuperacion_mensajes_pacientes} & Recuperación de mensajes de Pacientes & PASA \\
    \ref{cp:i:api:recuperacion_mensajes_cuidadores} & Recuperación de mensajes de Cuidadores correcta & PASA \\
    \ref{cp:i:api:recuperacion_paginas_mensajes_concreta} & Recuperación de páginas de mensajes concreta & PASA \\
    \ref{cp:i:api:recuperacion_mensajes_pagina_incorrecta} & Recuperación de mensajes con página incorrecta & PASA \\
    \ref{cp:i:api:recuperacion_mensajes_usuario_incompleto} & Recuperación de mensajes con usuario incompleto & PASA \\
    \ref{cp:i:api:recuperacion_mensajes_cuidador_incorrecta} & Recuperación de mensajes de Cuidador incorrecta & PASA \\
    \ref{cp:i:api:conexion_sala_mensajeria} & Conexión a la sala de mensajería & PASA \\
    \ref{cp:i:api:desconexion_sala_mensajeria} & Desconexión de la sala de mensajería & PASA \\
    \ref{cp:i:api:envio_mensaje_texto} & Envío de un mensaje de texto & PASA \\
    \ref{cp:i:api:envio_tarea} & Envío de una tarea & PASA \\ \hline
    \ref{cp:i:api:marcar_notificacion_valida_leida} & Marcar una notificación válida como leída & PASA \\
    \ref{cp:i:api:marcar_notificacion_invalida_leida} & Marcar una notificación inválida como leída & PASA \\
    \ref{cp:i:api:marcar_notificaciones_leidas} & Marcar todas las notificaciones como leídas & PASA \\
    \ref{cp:i:api:recuperacion_notificaciones_no_leidas_por_defecto} & Recuperación de notificaciones no leídas por defecto & PASA \\
    \ref{cp:i:api:recuperacion_notificaciones_no_leidas_edad_especificada} & Recuperación de notificaciones no leídas de edad especificada & PASA \\ \hline
    \ref{cp:i:api:creacion_tarea} & Creación de una tarea & PASA \\
    \ref{cp:i:api:creacion_tarea_invalida} & Creación de una tarea inválida & PASA \\
    \ref{cp:i:api:eliminacion_tarea_valida} & Eliminación de una tarea válida & PASA \\
    \ref{cp:i:api:eliminacion_tarea_invalida} & Petición a DELETE /task de una tarea válida & PASA \\
    \ref{cp:i:api:marcar_tarea_hecha} & Marcar tarea como hecha & PASA \\
    \ref{cp:i:api:marcar_tarea_no_hecha} & Marcar tarea como no hecha & PASA \\
    \ref{cp:i:api:marcar_tarea_hecha_hecha} & Marcar tarea hecha como hecha & PASA \\
    \ref{cp:i:api:marcar_tarea_no_hecha_no_hecha} & Marcar tarea no hecha como no hecha & PASA \\
    \ref{cp:i:api:recuperacion_tareas_paciente} & Recuperación de tareas de un Paciente & PASA \\
    \ref{cp:i:api:recuperacion_tareas_cuidador} & Recuperación de tareas de un Cuidador & PASA \\
    \ref{cp:i:api:recuperacion_tareas_edad_maxima} & Recuperación de tareas con edad máxima & PASA \\ \hline
    \caption{Segunda parte de los resultados de las pruebas de integración de la API}
\end{longtable}
\end{figure}

\begin{figure}[H]
\begin{longtable}{ l l c }
    \hline
    Referencia & Caso & Resultado \\
    \hline   
    \ref{cp:i:api:actualizacion_usuario} & Actualización de usuario & PASA \\
    \ref{cp:i:api:actualizacion_usuario_invalida} & Actualización de usuario inválida & PASA \\
    \ref{cp:i:api:creacion_vinculo} & Creación de vínculo & PASA \\
    \ref{cp:i:api:creacion_vinculo_invalida} & Creación de vínculo inválida & PASA \\
    \ref{cp:i:api:eliminacion_vinculo_valida} & Eliminación de vínculo válida & PASA \\
    \ref{cp:i:api:generacion_codigo_vinculacion} & Generación de código de vinculación & PASA \\
    \ref{cp:i:api:recuperacion_vinculos_paciente} & Recuperación de vínculos de un Paciente & PASA \\
    \ref{cp:i:api:recuperacion_vinculos_cuidador} & Recuperación de vínculos de un Cuidador & PASA \\
    \ref{cp:i:api:recuperacion_vinculos_usuario_incompleto} & Recuperación de vínculos de un usuario incompleto & PASA \\
    \ref{cp:i:api:recuperacion_paciente_cuidador_vinculado} & Recuperación de Paciente de un Cuidador vinculado & PASA \\
    \ref{cp:i:api:recuperacion_paciente_cuidador_no_vinculado} & Recuperación de Paciente de un Cuidador no vinculado & PASA \\
    \ref{cp:i:api:recuperacion_paciente_invalida} & Recuperación de Paciente inválida & PASA \\\hline
    \caption{Tercera parte de los resultados de las pruebas de integración de la API}
\end{longtable}
\end{figure}
\subsection{Aplicación móvil}

\begin{longtable}{ l l c }
    \hline
    Referencia & Caso & Resultado \\
    \hline
    \ref{cp:i:app:registro_paciente} & Registro de un Paciente en la aplicación & NO \\
    \ref{cp:i:app:registro_cuidador} & Registro de un Cuidador en la aplicación & NO \\ \hline
    \ref{cp:i:app:inicio_sesion} & Inicio de sesión en la aplicación & NO \\ \hline
    \ref{cp:i:app:mostrar_codigo_vinculacion} & Mostrar código de vinculación en la aplicación & NO \\
    \ref{cp:i:app:desplegar_escaner} & Desplegar escáner en la aplicación & NO \\
    \ref{cp:i:app:eliminiacion_vinculo_paciente} & Eliminación de vínculo de un Paciente & NO \\
    \ref{cp:i:app:eliminacion_vinculo_cuidador} & Eliminación de vínculo de un Cuidador & NO \\ \hline
    \ref{cp:i:app:compartir_ubicacion} & Compartir ubicación en la aplicación & NO \\ \hline
    \ref{cp:i:app:gestionar_tareas_tasks} & Gestionar las tareas desde Tareas & NO \\
    \ref{cp:i:app:gestionar_tareas_feed} & Gestionar las tareas desde el Feed & NO \\ \hline
    \ref{cp:i:app:enviar_recibir_mensajes} & Enviar y recibir mensajes en la aplicación & NO \\ \hline
    \ref{cp:i:app:gestionar_notificaciones} & Gestionar notificaciones en la aplicación & NO \\ \hline
    \ref{cp:i:app:consultar_vinculos} & Consultar los vínculos en la aplicación & NO \\ \hline
    \ref{cp:i:app:consultar_paciente} & Consultar paciente en la aplicación & NO \\ \hline
    \caption{Resultados de las pruebas de integración de la aplicación}
\end{longtable}

No se han podido llevar a cabo pruebas de integración en la aplicación. El equipo de desarrollo no ha sido capaz de lanzar la aplicación con Espresso y conseguir realizar el inicio de sesión a través de la autenticación de Google por problemas con las cuentas de usuario. Ha sido el mayor error del proyecto.

\section{Pruebas usabilidad y accesibilidad}

Por desgracia, \textbf{no fue posible} llevar a cabo estas pruebas. Todas las asociaciones contactadas rechazaron poder prestar ayuda y sin ellas, y con la situación de pandemia que ha existido durante el desarrollo, ha sido imposible contactar con usuarios potenciales de la aplicación para que respondiese a nuestro formulario de \fref{sec:usabilidad_accesibilidad}.