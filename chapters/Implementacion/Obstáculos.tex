\chapter{Obstáculos}
\label{ch:obstaculos}

Durante la creación del sistema se encontraron una serie de dificultades que supusieron un obstáculo en la implementación y ralentizaron el desarrollo. Aquí se exponen las más importantes.

\section{Comunicación cifrada}

Una de las primeras funciones agregadas al sistema fue el establecimiento de la comunicación entre la \acrshort{api} y la aplicación móvil. En esa primera implementación se utilizó un prototipo de cada subsistema que realizase la comunicación más básica posible tanto a través del WebSocket como por medio de \acrshort{http}. Esos primeros intercambios resultaron en errores debido a la \textbf{falta de encriptación} de las respuestas de la \acrshort{api}.

El siguiente paso en el intento de sobrepasar este problema fue el de lanzar Express como servidor \acrshort{https} con certificados generados en la máquina de ejecución local de dicho servidor. De cara a permitir esta comunicación dichos certificados debían ser también copiados en la aplicación móvil para añadirlos a una lista blanca de entidades confiables. Tras preparar todo, la comunicación siguió sin ser posible pues los certificados generados \textbf{no contaban con una fuente de confianza} y Android los rechazaba.

Finalmente, y viendo que generar un servidor \acrshort{https} con certificados emitidos por fuentes confiables escapaba al alcance del proyecto se decidió \textbf{modificar la configuración de red} de la aplicación y permitir el tráfico de texto plano. Para reducir el peligro de esta decisión al mínimo se realizó dicha permisión únicamente a las direcciones IP locales y del servidor a desplegar. Tras este cambio, que se puede ver en el fichero \code{network\textunderscore security\textunderscore config.xml} se pudo realizar envío de información entre ambos subsistemas.

\section{Ejecución en dispositivos móviles antes del despliegue}

A lo largo del desarrollo era posible realizar la comunicación entre la \acrshort{api} y la aplicación móvil durante su ejecución en la misma máquina local por medio de la especificación de la IP \code{http://10.0.2.2} para conseguir que la aplicación pudiese encontrar al servidor, pues es la dirección que representa a la máquina local sobre la que se ejecuta el emulador.

Sin embargo, esta dirección existe sólo como enrutamiento en los emuladores de \textbf{Android Studio}, pero no en los dispositivos móviles en los que se pueda ejecutar la aplicación, aunque sea a través de la conexión con el \acrshort{ide}. Es por esto, que a lo largo del desarrollo no fue posible probar la aplicación en un dispositivo móvil real hasta que se llevó a cabo el despliegue del servidor en la nube.

\section{Despliegue del servidor}

El despliegue del servidor en la nube no constó de un obstáculo sino que lo fue en sí mismo. Ante la falta de experiencia particular en despliegues de este tipo o en ingeniería DevOps la tarea resultó de \textbf{gran dificultad} y consumió una cantidad de tiempo mucho mayor a la esperada inicialmente. Dilatada especialmente por el propio tiempo que llevaba cada despliegue.

De cara a poder lograr este despliegue en la nube fue necesario realizar una serie de cambios en el código del servidor que permitiesen que la compilación, ejecución y lanzamiento fuesen fructíferos. Se encontraron problemas en cada uno de estos pasos, los más resaltables fueron: 
\begin{itemize}
    \item A la hora de compilar se generaron rutas entre ficheros erróneas en la generación del código Javascript a partir de los alias de TypeScript, esto llevó a la necesidad de instalar librerías adicionales que cubriesen este error.
    \item La ejecución de los archivos compilados también falló puesto que la acción de despliegue generada por Azure intentaba ejecutar el archivo equivocado.
    \item Problemas al encontrar variables de entorno.
    \item Incapacidad de conexión con la base de datos por el listado blanco de direcciones IP.
\end{itemize}

Finalmente, el despliegue fue posible con todos los pasos indicados en el \nameref{ch:manual_despliegue} (\fref{ch:manual_despliegue}), pero esta tarea consumió más tiempo del esperado en la \nameref{ch:planificacion_temporal} (\fref{ch:planificacion_temporal}).