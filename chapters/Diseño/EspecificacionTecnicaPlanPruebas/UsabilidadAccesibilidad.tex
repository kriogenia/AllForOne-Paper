\section{Pruebas de usabilidad y accesibilidad}
\label{sec:usabilidad_accesibilidad}

Para la realización de estas pruebas se contactarán con familias afectadas por Alzheimer que puedan probar la aplicación de forma real y dar su veredicto acerca de ella por medio de cuestionario. Posteriormente, se analizarán las preguntas del cuestionario y se realizarán las mejores necesarias en función de la información recabada.

Existirán dos grupos de control dentro de estas pruebas. Por un lado, tendremos los usuarios que padecen la enfermedad (Pacientes); y por otro lado, los usuarios que se vincularán con estos, los Cuidadores. Debido a los ratios de los dos grupos de población se estima obtener una muestra mayor de Cuidadores que de Pacientes, sin embargo, la opinión de los Pacientes será tomada más en cuenta pues son los usuarios principales.

Las preguntas del cuestionario podrán ser de diferentes tipos: \textbf{selección única}, preguntas que requieren la selección de una de las opciones elegidas; \textbf{rellenar}, preguntas que solicitan una respuesta redactada, el tamaño puede variar; \textbf{rangos}, se response selección un único punto del rango establecido, por ejemplo, un rango de cinco niveles desde \emph{Nada satisfecho} a \emph{Muy satisfecho}; o \textbf{elección múltiple}, donde el usuario podrá seleccionar cuantas opciones quiera entre las ofrecidas. Además, las preguntas podrán ser o no opcionales.\newline

\textbf{Pregunta 1}: ¿Eres Paciente o Cuidador?\newline
\textbf{Tipo}: Selección única\newline
\textbf{Opciones}: Paciente, Cuidador\newline
\textbf{Obligatoria}

\textbf{Pregunta 2}: ¿Usas habitualmente aplicaciones móviles?\newline
\textbf{Tipo}: Rango, cinco opciones \newline
\textbf{Opciones}: Desde Nunca a Muy habitualmente\newline
\textbf{Obligatoria}

\textbf{Pregunta 3}: Si eres Paciente, ¿con cuántos Cuidadores te has vinculado?\newline
\textbf{Tipo}: Rango, siete opciones\newline
\textbf{Opciones}: De 0 a 6\newline
\emph{Opcional}

\textbf{Pregunta 4}: ¿Has tenido dificultades para crear tu cuenta?\newline
\textbf{Tipo}: Selección única \newline
\textbf{Opciones}: Sí, No\newline
\textbf{Obligatoria}

\textbf{Pregunta 5}: En caso afirmativo, ¿cuáles?\newline
\textbf{Tipo}: Rellenar, línea\newline
\emph{Opcional}

\textbf{Pregunta 6}: ¿Has tenido dificultades para vincularte?\newline
\textbf{Tipo}: Selección única \newline
\textbf{Opciones}: Sí, No\newline
\textbf{Obligatoria}

\textbf{Pregunta 7}: En caso afirmativo, ¿cuáles?\newline
\textbf{Tipo}: Rellenar, línea\newline
\emph{Opcional}

\textbf{Pregunta 8}: ¿Qué datos de contacto has añadido a tu perfil?\newline
\textbf{Tipo}: Selección múltiple \newline
\textbf{Opciones}: Teléfono principal, Teléfono secundario, Dirección postal, Email
\textbf{Obligatoria}

\textbf{Pregunta 9}: ¿Echas en falta algún dato de contacto que crees que podría ser útil?\newline
\textbf{Tipo}: Rellenar, línea\newline
\emph{Opcional}

\textbf{Pregunta 10}: ¿Has contactado a algún otro usuarios a través de los datos de su tarjeta?\newline
\textbf{Tipo}: Selección única \newline
\textbf{Opciones}: Sí, No\newline
\textbf{Obligatoria}

\newpage

\textbf{Pregunta 11}: En caso negativo, ¿sabías que se podía?\newline
\textbf{Tipo}: Selección única \newline
\textbf{Opciones}: Sí, No\newline
\emph{Opcional}

\textbf{Pregunta 12}: ¿Qué funciones de la aplicación has utilizado?\newline
\textbf{Tipo}: Selección múltiple \newline
\textbf{Opciones}: Vínculos, Feed, Tareas, Localización, Notificaciones, Ninguna
\textbf{Obligatoria}

\textbf{Pregunta 13}: ¿Qué funciones añadirías a la aplicación?\newline
\textbf{Tipo}: Rellenar, área de texto\newline
\emph{Opcional}

\textbf{Pregunta 14}: Valora la utilidad de las siguientes funciones.\newline
\textbf{Tipo}: Rango, cinco opciones\newline
\textbf{Opciones}: Desde Nada a Mucho, y Ninguno\newline
\textbf{Cuestiones}: Vínculos, Feed, Tareas, Localización, Notificaciones
\textbf{Obligatoria}

\textbf{Pregunta 15}: Valora la facilidad de uso de las siguientes funciones.\newline
\textbf{Tipo}: Rango, cinco opciones\newline
\textbf{Opciones}: Desde Nada a Mucho, y Ninguno\newline
\textbf{Cuestiones}: Contacto de vínculos, Envío de mensajes, Creación de tareas, Cambio de estado de tareas, Eliminación de tareas, Compartir ubicación\newline
\textbf{Obligatoria}

\textbf{Pregunta 16}: Valora la legibilidad de las siguientes pantallas.\newline
\textbf{Tipo}: Rango, cinco opciones\newline
\textbf{Opciones}: Desde Nada a Mucho, y Ninguno\newline
\textbf{Cuestiones}: Inicio, Principal, Vínculos, Feed, Tareas, Localización, Notificaciones, Ajustes\newline
\textbf{Obligatoria}

\textbf{Pregunta 17}: ¿Te ha resultado útil la aplicación?\newline
\textbf{Tipo}: Selección única \newline
\textbf{Opciones}: Sí, No\newline
\textbf{Obligatoria}

\newpage

\textbf{Pregunta 18}: Selecciona tu grado de satisfacción.\newline
\textbf{Tipo}: Rango, diez opciones\newline
\textbf{Opciones}: De 1 a 10\newline
\textbf{Obligatoria}

\textbf{Pregunta 19}: Opiniones adicionales\newline
\textbf{Tipo}: Rellenar, área de texto\newline
\emph{Opcional}