\subsection{RSU. Requisitos de sesión de usuario}

\begin{enumerate}[label*=RSU \arabic*.]
    \item Los usuarios podrán iniciar sesión en la aplicación mediante autenticación con su cuenta de Google.
    \item Los inicios de sesión serán validados con una petición a la \acrshort{api} \acrshort{rest} con el \gls{token} obtenido de la validación con Google.
    \begin{enumerate}[label*=\arabic*.]
        \item Si el inicio de sesión es inválido el usuario será notificado del error y se le permitirá reintentarlo.
        \item \label{req:inicio_sesion} Si el inicio de sesión es válido y el usuario ya está registrado, se creará la sesión del usuario.
        \begin{enumerate}[label*=\arabic*.]
            \item \label{req:sesion} La \acrshort{api} \acrshort{rest} creará y almacenará la sesión del usuario. Dicha sesión estará compuesta por:
            \begin{enumerate}[label*=\arabic*.]
                \item \label{req:token_sesion}  \Gls{token} de sesión para realizar las peticiones. Tendrá un tiempo de validez de TIEMPO\textunderscore VALIDEZ\textunderscore TOKEN\textunderscore SESIÓN.
                \item \label{req:token_refresco} \Gls{token} de refresco de sesión para mantener la sesión activa y obtener un nuevo par de \glspl{token}. Tendrá un tiempo de validez de TIEMPO\textunderscore VALIDEZ\textunderscore TOKEN\textunderscore REFRESCO.
            \end{enumerate}
            \item La sesión pasará a ser inválida cuando caduque su \labelcref{req:token_refresco}.
            \item \label{req:info_sesion} La \acrshort{api} \acrshort{rest} enviará a la aplicación la información de sesión del usuario.
            \begin{enumerate}[label*=\arabic*.]
                \item \Gls{token} de sesión (\labelcref{req:token_sesion}).
                \item \Gls{token} de refresco de sesión (\labelcref{req:token_refresco}).
                \item Instante de caducidad de la sesión.
            \end{enumerate}
            \item El usuario será redirigido a la pantalla principal de la aplicación.
        \end{enumerate}
        \item \label{req:registro} Si el inicio de sesión es válido y el usuario es nuevo se le redirigirá a la actividad de creación de usuarios.
        \begin{enumerate}[label*=\arabic*.]
            \item \label{req:nombre_usuario} El usuario deberá indicar un nombre válido.
            \item \label{req:rol_usuario} El usuario deberá seleccionar un rol entre los posibles:
            \begin{enumerate}[label*=\arabic*.]
                \item Paciente
                \item Cuidador
            \end{enumerate}
            \item \label{req:info_adicional} El usuario podrá añadir información adicional a su perfil.
            \begin{itemize}
                \item Teléfono de contacto
                \item Teléfono de contacto alternativo
                \item Dirección de correo electrónico
                \item Dirección física. Conformada por dos campos para componer la dirección postal, un campo para la localidad y un campo para la región.
            \end{itemize}
            \item La aplicación pedirá confirmación al usuario de los datos insertados.
            \begin{enumerate}[label*=\arabic*.]
                \item Si el usuario confirma los datos insertados, los datos serán enviados a la \acrshort{api} \acrshort{rest} para su almacenamiento y validación.
                \item Si el usuario no es válido, la \acrshort{api} \acrshort{rest} responderá con un mensaje de error.
                \item Si el usuario es válido, la \acrshort{api} \acrshort{rest} responderá a la petición con un mensaje de éxito y el perfil completo del usuario.
                \item Tras la creación exitosa del perfil, el usuario será dado de alta siguiendo \labelcref{req:inicio_sesion}.
            \end{enumerate}
        \end{enumerate}
        \item La sesión podrá ser renovada con una solicitud a la \acrshort{api} \acrshort{rest} con el \gls{token} de refresco de sesión mientras este siga siendo válido.
            \begin{enumerate}[label*=\arabic*.]
                \item Refrescar una sesión generará una tupla como la de \labelcref{req:sesion} y sustituirá la almacenada anteriormente.
                \item La \acrshort{api} \acrshort{rest} responderá con la información de la sesión de usuario como en \labelcref{req:info_sesion}.
                \item Al refrescar una sesión los \glspl{token} anteriores quedarán invalidados.
            \end{enumerate}
    \end{enumerate}
    \item \label{req:cierre_sesion} Los usuarios con la sesión iniciada podrán cerrar sesión en la aplicación.
    \begin{enumerate}[label*=\arabic*.]
        \item La aplicación enviará la petición de cierre de sesión a la \acrshort{api} \acrshort{rest} con el \gls{token} de autenticación.
        \item La \acrshort{api} \acrshort{rest} eliminará la sesión almacenada y confirmará la acción a la aplicación.
        \item Si el cierre de sesión es exitoso el usuario será redirigido a la pantalla de inicio de sesión.
    \end{enumerate}
\end{enumerate}

\subsection{RGU. Requisitos de gestión de usuarios}

\begin{enumerate}[label*=RGU \arabic*.]
    \item \label{req:consultar_info_paciente} Los usuarios podrán ver la información del Paciente vinculado en la pantalla principal.
    \begin{enumerate}[label*=\arabic*.]
        \item La información a mostrar será suministrada por la \acrshort{api} \acrshort{rest}.
        \item En el caso de Pacientes se les mostrará su propia información.
        \item La información mostrada será:
        \begin{itemize}
            \item Nombre.
            \item Teléfono de contacto.
            \item Teléfono de contacto alternativo.
            \item Dirección de correo electrónico.
            \item Dirección postal.
        \end{itemize}
        \item \label{req:accion_contactos} Los datos de contacto de la tarjeta proporcionarán atajos para usarlos:
        \begin{enumerate}[label*=\arabic*.]
            \item Los números de teléfono desplegarán el dial del teléfono con el número marcado.
            \item El correo electrónico abrirá una aplicación de correo electrónico con un nuevo correo preparado para ser enviado a la dirección.
            \item La dirección postal desplegará Google Maps con la dirección ya buscada.
        \end{enumerate}
    \end{enumerate}
    \item Los usuarios podrán actualizar su información.
    \begin{enumerate}[label*=\arabic*.]
        \item El usuario actualizará los campos que prefiera de:
        \begin{itemize}
            \item Nombre.
            \item Teléfono de contacto.
            \item Teléfono de contacto alternativo.
            \item Dirección de correo electrónico.
            \item Dirección postal.
        \end{itemize}
        \item Cuando el usuario confirme la actualización, los datos serán enviados a la \acrshort{api} \acrshort{rest} para validación.
        \item La \acrshort{api} \acrshort{rest} confirmará los cambios efectuados y responderá con la información completa del usuario.
        \item La aplicación se actualizará con los nuevos datos del usuario.
        \item \label{req:noti_actualizar} Una notificación será enviada a los usuarios vinculados acerca de la actualización de los datos del usuario.
    \end{enumerate}
    \item La aplicación permitirá a los usuarios establecer vínculos con otros usuarios.
    \begin{enumerate}[label*=\arabic*.]
        \item Los Pacientes podrán vincular MAX\textunderscore VINCULOS\textunderscore PACIENTE usuarios de tipo Cuidador.
        \item Los Cuidadores podrán vincular MAX\textunderscore VINCULOS\textunderscore CUIDADOR Pacientes.
        \item \label{req:vinculo_paciente} Los Pacientes podrán generar un código QR para vincularse.
        \begin{enumerate}[label*=\arabic*.]
            \item El código QR representará un token provisto por la \acrshort{api} \acrshort{rest}.
            \item El token caducará en TIEMPO\textunderscore CADUCIDAD\textunderscore TOKEN\textunderscore VINCULACIÓN.
        \end{enumerate}
        \item \label{req:vinculo_cuidador} Los Cuidadores podrán escanear el código QR de un Paciente para completar el vínculo.
        \begin{enumerate}[label*=\arabic*.]
            \item El código leído será enviado a la \acrshort{api} \acrshort{rest}.
            \item La \acrshort{api} \acrshort{rest} validará el token y que el vínculo sea válido.
            \begin{enumerate}[label*=\arabic*.]
                \item Si el vínculo no es válido, una respuesta de error será emitida
                \item Si el vínculo es válido, se creará el vínculo entre ambos usuarios.
            \end{enumerate}
        \end{enumerate}
        \item \label{req:noti_nuevo_vinculo} La \acrshort{api} \acrshort{rest} emitirá una notificación de la creación del vínculo al resto de usuarios asociados.
    \end{enumerate}
    \item \label{req:borrar_vinculo} Los usuarios con vínculos activos podrán eliminar sus vínculos actuales a través de la aplicación.
    \begin{enumerate}[label*=\arabic*.]
        \item La petición se realizará con una llamada a la \acrshort{api} \acrshort{rest}.
        \item La \acrshort{api} \acrshort{rest} validará la petición y enviará una respuesta de confirmación.
        \item La \acrshort{api} \acrshort{rest} emitirá una notificación de la eliminación del vínculo al resto de usuarios asociados antes de la eliminación del vínculo.
    \end{enumerate}
    \item \label{req:consultar_info_cuidador} La aplicación permitirá listar en la aplicación los Cuidadores relacionados de los usuarios.
    \begin{enumerate}[label*=\arabic*.]
        \item La aplicación listará a los Pacientes sus Cuidadores vinculados.
        \item La aplicación listará a los Cuidadores el resto de Cuidadores vinculados por su Paciente.
        \item El listado de Cuidadores será proporcionado por una petición a la \acrshort{api} \acrshort{rest}.
        \item Con cada Cuidador listado se mostrará su información de contacto:
        \begin{itemize}
            \item Teléfono de contacto.
            \item Teléfono de contacto alternativo.
            \item Dirección de correo electrónico.
            \item Dirección postal.
        \end{itemize}
        \item Desde el listado, la aplicación permitirá a los usuarios los mismos atajos de funcionalidad que \labelcref{req:accion_contactos}.
    \end{enumerate}
\end{enumerate}

\subsection{RGT. Requisitos de gestión de tareas}

\begin{enumerate}[label*=RGT \arabic*.]
    \item \label{req:crear_tarea} Los Pacientes y sus Cuidadores podrán crear una nueva tarea desde la aplicación.
    \begin{enumerate}[label*=\arabic*.]
        \item Las tareas contarán con la siguiente información:
        \begin{enumerate}[label*=\arabic*.]
            \item Un identificador único, obligatorio y generado en el proceso de persistencia de la tarea.
            \item Título, obligatorio.
            \item Creador, obligatorio y rellenado automáticamente por la aplicación.
            \item Descripción, opcional.
            \item Instante de creación, obligatorio y rellenado automáticamente por la aplicación.
            \item Instante de la última actualización, obligatorio y por defecto será creada con el mismo valor que el instante de creación.
            \item Estado de la tarea. Puede ser completa o incompleta. Obligatorio, por defecto será creada como incompleta.
        \end{enumerate}
        \item Las tareas creadas en la aplicación se enviarán a la \acrshort{api} \acrshort{rest} para su validación y persistencia.
        \begin{enumerate}[label*=\arabic*.]
            \item Si la tarea es inválida, la \acrshort{api} \acrshort{rest} emitirá un mensaje de error y no persistirá la tarea.
            \item Si la tarea es válida, la \acrshort{api} \acrshort{rest} responderá con un mensaje de éxito con la información completa de la tarea creada.
        \end{enumerate}
        \item La creación de una tarea la añadirá a la lista de tareas del Paciente involucrado.
        \item \label{req:noti_nueva_tarea} El sistema emitirá una notificación de la tarea creada a los usuarios asociados.
    \end{enumerate}
    \item \label{req:listar_tarea} Los Pacientes y sus Cuidadores podrán consultar las tareas existentes en la aplicación.
    \begin{enumerate}[label*=\arabic*.]
        \item \label{req:tarea_relevancia} La lista de tareas será suministrada por la \acrshort{api} \acrshort{rest}. Se listarán todas las tareas del paciente que se consideren relevantes. Una tarea es relevante si cumple alguna de las siguientes condiciones:
        \begin{enumerate}[label*=\arabic*.]
            \item Si la tarea está incompleta.
            \item Si la última actualización de la tarea entra dentro del margen de días especificado por el usuario.
        \end{enumerate}
        \item \label{req:info_tarea} Las tareas listadas mostrarán el título, el creador, la descripción, el estado y las fechas de creación y actualización.
        \item Las tareas completas e incompletas serán visualmente diferenciables.
        \item Por defecto, las tareas aparecerán listadas de más recientes a más antiguas.
        \item Las tareas se podrán gestionar desde el listado.
    \end{enumerate}
    \item \label{req:marcar_tarea_hecha} Los Pacientes y sus Cuidadores podrán marcar una tarea del Paciente como hecha desde la aplicación.
    \begin{enumerate}[label*=\arabic*.]
        \item El usuario deberá confirmar que quiere marcar la tarea como hecha.
        \item La petición de actualización se hará a la \acrshort{api} \acrshort{rest}, que la validará.
        \begin{enumerate}[label*=\arabic*.]
            \item Si la petición no es válida, la \acrshort{api} \acrshort{rest} emitirá un mensaje de error.
            \item Si la petición es válida, la \acrshort{api} \acrshort{rest} emitirá una respuesta de confirmación.
        \end{enumerate}
        \item La tarea se actualizará visualmente para reflejar su nuevo estado.
        \item La tarea se seguirá mostrando aunque pierda relevancia (véase \labelcref{req:tarea_relevancia}) hasta que la lista sea refrescada. Permitiendo deshacer la acción.
        \item \label{req:noti_tarea_hecha} El sistema notificará el cambio en el estado a los usuarios asociados.
    \end{enumerate}
    \item \label{req:marcar_tarea_no_hecha} Los Pacientes y sus Cuidadores podrán marcar una tarea como no hecha desde la aplicación.
    \begin{enumerate}[label*=\arabic*.]
        \item El usuario deberá confirmar que quiere desmarcar la tarea como hecha.
        \item La petición de actualización se hará a la \acrshort{api} \acrshort{rest}, que la validará.
        \begin{enumerate}[label*=\arabic*.]
            \item Si la petición no es válida, la \acrshort{api} \acrshort{rest} emitirá un mensaje de error.
            \item Si la petición es válida, la \acrshort{api} \acrshort{rest} emitirá una respuesta de confirmación.
        \end{enumerate}
        \item La tarea se actualizará visualmente para reflejar su nuevo estado.
        \item \label{req:noti_tarea_no_hecha} El sistema notificará el cambio en el estado a los usuarios asociados.
    \end{enumerate}
    \item \label{req:eliminar_tarea} Los usuarios podrán eliminar tareas en la aplicación.
    \begin{enumerate}[label*=\arabic*.]
        \item Los Pacientes podrán eliminar cualquiera de sus tareas.
        \item Los Cuidadores sólo podrán eliminar las tareas creadas por ellos.
        \item El usuario deberá confirmar que quiere eliminar la tarea.
        \item La petición de eliminación se hará a la \acrshort{api} \acrshort{rest}, que la validará.
        \begin{enumerate}[label*=\arabic*.]
            \item Si la petición no es válida, la \acrshort{api} \acrshort{rest} emitirá un mensaje de error.
            \item Si la petición es válida, la \acrshort{api} \acrshort{rest} emitirá una respuesta de confirmación.
        \end{enumerate}
        \item \label{req:noti_tarea_eliminada} El sistema notificará a los usuarios vinculados de la eliminación de la tarea.
    \end{enumerate}
\end{enumerate}

\subsection{RSV. Requisitos de servicio de mensajería}

\begin{enumerate}[label*=RSV \arabic*.]
    \item La aplicación ofrecerá una sala de mensajería a los usuarios vinculados.
    \item La aplicación se conectará a un WebSocket de la \acrshort{api} de la sala.
    \item Los participantes de la sala de mensajería serán el Paciente y todos sus Cuidadores vinculados.
    \item Los mensajes de la sala pueden ser de tres tipos:
    \begin{enumerate}[label*=\arabic*.]
        \item Mensajes de texto.
        \item Tareas.
        \item Notificaciones.
    \end{enumerate}
    \item \label{req:listar_mensaje} La sala de mensajería mostrará, a los usuarios que se conecten, los mensajes más recientes.
    \begin{enumerate}[label*=\arabic*.]
        \item Los mensajes al abrir la sala serán servidos por la \acrshort{api} \acrshort{rest}.
        \item La \acrshort{api} \acrshort{rest} servirá los CANTIDAD\textunderscore GRUPO\textunderscore MENSAJES mensajes más recientes.
        \item Los participantes podrán cargar mensajes más antiguos.
        \item Al solicitar mensajes más antiguos la \acrshort{api} \acrshort{rest} servirá los CANTIDAD\textunderscore GRUPO\textunderscore MENSAJES mensajes más recientes siguientes.
        \item La aplicación mostrará los nuevos mensajes entrantes en tiempo real.
        \item Los mensajes entrantes serán servidos a través del WebSocket.
        \item Los mensajes salientes y entrantes serán visualmente diferentes.
        \item Los tipos de mensajes será visualmente diferentes.
        \item Los mensajes serán agrupados por fecha de emisión.
    \end{enumerate}
    \item \label{req:envio_mensaje} Los participantes de una sala de mensajería podrán enviar mensajes de texto al resto de pacientes.
    \begin{enumerate}[label*=\arabic*.]
        \item El mensaje saliente se enviará a través del WebSocket.
        \item La \acrshort{api} persistirá el mensaje y lo reenviará por el WebSocket al resto de participantes.
        \item El mensaje saliente se añadirá a la lista de mensajes mostrados.
    \end{enumerate}
    \item Los participantes de una sala de mensajería podrán gestionar tareas a través de la sala.
    \begin{enumerate}[label*=\arabic*.]
        \item Las tareas serán tratadas como mensajes.
        \item Los participantes podrán crear una tarea que se enviará por la sala de mensajería igual que en \labelcref{req:envio_mensaje}.
        \item Las tareas creadas de esta forma cumplirán las mismas pautas que en \labelcref{req:crear_tarea}
        \item Las tareas mostradas en la sala de mensajería podrán ser marcadas como hechas como en \labelcref{req:marcar_tarea_hecha}
        \item Las tareas mostradas en la sala de mensajería podrán ser desmarcadas como hechas como en \labelcref{req:marcar_tarea_no_hecha}
        \item Las tareas mostradas en la sala de mensajería podrán ser eliminadas como en \labelcref{req:eliminar_tarea}
        \begin{enumerate}[label*=\arabic*.]
            \item Las tareas eliminadas mientras el usuario esté conectado serán eliminadas de la lista.
            \item La eliminación de una tarea por cualquier medio será comunicada a los usuarios conectados a la sala con una notificación.
        \end{enumerate}
        \item Las acciones sobre las tareas de la sala serán enviadas a la API a través del WebSocket.
        \item La \acrshort{api} ejecutará la petición y enviará un mensaje al resto de participantes de la sala que actualice la tarea.
    \end{enumerate}
    \item Los participantes serán desconectados del WebSocket de la sala al abandonarla.
    \item La sala de mensajería mostrará notificaciones de la conexión y desconexión de usuarios de la sala al resto de usuarios conectados.
\end{enumerate}

\subsection{RSG. Requisitos del sistema de geolocalización}

\begin{enumerate}[label*=RSG \arabic*.]
    \item La aplicación servirá un mapa con un servicio de geolocalización mutua entre usuarios conectados.
    \item La aplicación utilizará la \acrshort{api} de Google Maps.
    \item \label{req:compartir_ubicacion} A través de la aplicación los usuarios podrán compartir su ubicación.
    \begin{enumerate}[label*=\arabic*.]
        \item La aplicación solicitará permiso al usuario para utilizar el \acrshort{gps} integrado en el teléfono móvil.
        \begin{enumerate}[label*=\arabic*.]
            \item Si el usuario deniega acceso al \acrshort{gps}, un mensaje de error será mostrado y la función no se llevará a cabo.
            \item Si el usuario permite el acceso, se empezará a compartir la ubicación y se mostrará el mapa.
        \end{enumerate}
        \item La aplicación se conectará a un WebSocket compartido con el resto de usuarios relacionados.
        \item La ubicación comenzará a compartirse automáticamente al iniciar el servicio de geolocalización.
        \item La ubicación del usuario será enviada a la \acrshort{api} través del WebSocket en intervalos regulares de INTERVALO\textunderscore COMPARTIR\textunderscore UBICACIÓN.
        \item La posición actual del usuario se mostrará en un mapa.
        \item Cada actualización de la ubicación del usuario modificará su posición visual en el mapa.
        \item \label{req:noti_ubicacion_empezar} Cuando un usuario empiece a compartir su ubicación, una notificación será enviada al resto de usuarios conectados.
    \end{enumerate}
    \item Los usuarios compartiendo ubicación podrán dejar de hacerlo abandonando el mapa del servicio.
    \begin{enumerate}[label*=\arabic*.]
        \item Si un usuario abandona el servicio, la aplicación se desconectará del WebSocket.
        \item\label{req:noti_ubicacion_parar}  Cuando un usuario se desconecta del servicio, una notificación será enviada al resto de usuarios conectados.
        \item La notificación de un usuario dejando de compartir su ubicación eliminará la notificación previa del comienzo de la función.
    \end{enumerate}
    \item \label{req:ver_ubicaciones} Los usuarios conectados al servicio podrán ver la ubicación de otros usuarios conectados.
    \begin{enumerate}[label*=\arabic*.]
        \item Las ubicaciones ajenas se recibirán en tiempo real.
        \item Las ubicaciones del resto de usuarios se recibirán a través del WebSocket.
        \item La aplicación mostrará un aviso en pantalla cuando la lista de usuarios conectados al servicio cambie. 
        \begin{enumerate}[label*=\arabic*.]
            \item Cuando un usuario entre al servicio.
            \item Cuando un usuario abandone el servicio.
        \end{enumerate}
        \item Las ubicaciones del resto de usuarios se mostrará con un marcador diferenciado.
        \begin{enumerate}[label*=\arabic*.]
            \item El marcador contendrá la información del usuario respectivo.
            \item La actualización de las ubicaciones desplazará el marcador asociado al usuario a la nueva ubicación.
            \item Cuando un usuario se desconecte, su marcador será retirado del mapa.
        \end{enumerate}
    \end{enumerate}
    \item Los usuarios podrán desplazarse por el mapa.
    \item Los usuarios podrán hacer acercar o alejar la vista en el mapa.
    \item Los usuarios podrán centrar el mapa sobre su ubicación.
\end{enumerate}

\subsection{RSN: Requisitos del sistema de notificaciones}

\begin{enumerate}[label*=RSN \arabic*.]
    \item \label{req:notificaciones} La aplicación notificará a los usuarios acerca de diversos sucesos de interés.
    \begin{enumerate}[label*=\arabic*.]
        \item Un nuevo vínculo genera un nuevo usuario asociado (véase \labelcref{req:noti_nuevo_vinculo})
        \item Un usuario asociado ha actualizado sus datos personales (véase \labelcref{req:noti_actualizar})
        \item Un usuario asociado ha creado una nueva tarea (véase \labelcref{req:noti_nueva_tarea}).
        \item Un usuario asociado ha eliminado una tarea (véase \labelcref{req:noti_tarea_eliminada}).
        \item Un usuario asociado ha marcado una tarea como hecha (véase \labelcref{req:noti_tarea_hecha}).
        \item Un usuario asociado ha desmarcado una tarea como hecha (véase \labelcref{req:noti_tarea_no_hecha}).
        \item Un usuario asociado ha comenzado a compartir su ubicación (véase \labelcref{req:noti_ubicacion_empezar}).
        \item Un usuario asociado ha dejado de compartir su ubicación (véase \labelcref{req:noti_ubicacion_parar}).
    \end{enumerate}
    \item La aplicación se unirá a un WebSocket para recibir las notificaciones.
    \item Las notificaciones pueden ser pendientes o no, dependiendo de si el usuario las ha leído.
    \item La aplicación mostrará la cantidad de notificaciones pendientes del usuario.
    \item \label{req:consultar_notificaciones} El usuario podrá ver en la aplicación sus notificaciones pendientes:
    \begin{enumerate}[label*=\arabic*.]
        \item La lista de notificaciones será servida por la \acrshort{api} \acrshort{rest}.
        \item Las notificaciones aparecerán agrupadas por fecha de emisión.
        \item \label{req:leer_notificaciones} Las notificaciones podrán ser marcadas como leídas individualmente o todas a la vez.
        \item Las notificaciones referentes a funciones de la aplicación podrán ser usadas para navegar a dicha función.
    \end{enumerate}
\end{enumerate}

\subsection{RNF: Requisitos no funcionales}

\begin{enumerate}[label*=RNF \arabic*.]
    \item La aplicación estará disponible en Android 8.1 y versiones posteriores del sistema operativo.
    \item La aplicación estará disponible en los siguientes idiomas:
    \begin{itemize}
        \item Inglés
        \item Español
        \item Gallego
    \end{itemize}
    \item La \acrshort{api} debe estar alojada en la nube.
    \begin{enumerate}[label*=\arabic*.]
        \item El sistema de la nube de alojamiento debe ser Microsoft Azure.
        \item La disponibilidad de la API debe ser superior al 99\%. Esto corresponde a una disponibilidad de 8672 horas y 24 minutos por año.
        \item La velocidad de respuesta a peticiones de la \acrshort{api} no debe exceder los TIEMPO\textunderscore MAXIMO\textunderscore RESPUESTA\textunderscore API.
    \end{enumerate}
    \item La base de datos debe estar alojada en la nube.
    \begin{enumerate}[label*=\arabic*.]
        \item El sistema de base de datos en la nube debe ser MongoDB Atlas.
        \item La disponibilidad de la base de datos debe ser superior al 99\%. Esto corresponde a una disponibilidad de 8672 horas y 24 minutos por año.
    \end{enumerate}
    \item El sistema conectará con las siguientes \acrshort{api}s de terceros:
    \begin{enumerate}[label*=\arabic*.]
        \item Google OAuth 2.0.
        \item Google Maps for Android.
    \end{enumerate}
    \item Los usuarios únicamente necesitarán un conocimiento mínimo en el manejo de móviles inteligentes.
\end{enumerate}

\subsection{Valores de requisitos}

\begin{table}[H]
    \centering
    \begin{tabular}{l|l}
    ID & Valor \\ \hline 
    CANTIDAD\textunderscore GRUPO\textunderscore MENSAJES & 30 \\ 
    INTERVALO\textunderscore COMPARTIR\textunderscore UBICACIÓN & 5 segundos \\  
    MAX\textunderscore VINCULOS\textunderscore CUIDADOR & 1 \\
    MAX\textunderscore VINCULOS\textunderscore PACIENTE & 6 \\     
    TIEMPO\textunderscore MAXIMO\textunderscore RESPUESTA\textunderscore API & 10 segundos \\ 
    TIEMPO\textunderscore VALIDEZ\textunderscore TOKEN\textunderscore SESIÓN & 1 hora \\ 
    TIEMPO\textunderscore VALIDEZ\textunderscore TOKEN\textunderscore REFRESCO & 24 horas \\
    TIEMPO\textunderscore VALIDEZ\textunderscore TOKEN\textunderscore VINCULACION & 1 minuto
    \end{tabular}
    \caption{Valores especificados en requisitos}
\end{table}

