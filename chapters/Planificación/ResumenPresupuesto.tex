\chapter{Resumen del presupuesto}
\label{ch:resumen_presupuesto}

El desarrollo de este proyecto \textbf{carece de perspectiva comercial o de negocio}. No existe ningún porcentaje de beneficio aplicado sobre el coste del proyecto, no se aplica el impuesto de valor añadido y el margen entre el coste total y la facturación se ha reducido a un escueto 1.89\%.

El equipo de desarrollo cuenta con \textbf{un único miembro} y en base a ello se ha considerado el desarrollo como un proyecto \emph{freelance} sin declaración de autónomo. A pesar de existir un único miembro encargado del desarrollo, los costes de las diferentes partidas y tareas se realizaron teniendo en cuenta el cargo que se personará en cada caso concreto en vez de utilizar un único salario y rol general. Los salarios utilizados están basados en las medias encontradas en el portal web Glassdoor\footnote{https://www.glassdoor.es/member/home/index.htm}, con una reducción compensatoria de la inexperiencia del desarrollador.

De cara al cálculo de costes indirectos y de medios de producción del equipo de desarrollo se tomaron en cuenta: el cálculo estimado de consumo eléctrico del portátil usado en el desarrollo, una única licencia de pago\footnote{Licencia de Office 365 para la elaboración del presupuesto con Microsoft Excel} puesto que el resto de softwares utilizados son gratuitos, el coste del despliegue del servidor de prueba durante el desarrollo y el coste por hora en base a la amortización de los teléfonos y ordenador empleados durante el desarrollo y que ya se poseían de antes.

A continuación se ofrece el resumen del presupuesto. El desglose detallado de este y sus partidas se puede encontrar en el \fref{ch:anexo_presupuesto} \nameref{ch:anexo_presupuesto} y en la hoja de cálculo adjunta a este documento y que se indica en el \fref{ch:documentacion_adicional}. Cada partida listada en este resumen contiene también la referencia de su partida detallada en dicho anexo.

\vspace{5pt}
% Presupuesto de costes expandido
\begin{longtable}{ l l l r }
    \hline
    Código & Partida & Desglose & Importe \\
    \hline
    1 & Planificación                                   & \fref{pre:planificacion}  & 1 149.00€ \\
    2 & Análisis                                        & \fref{pre:analisis}       & 1 174.00€ \\
    3 & Diseño                                          & \fref{pre:diseno}         & 1 500.00€ \\
    4 & Construcción                                    & \fref{pre:construccion}   & 6 957.00€ \\
    5 & Formación                                       & \fref{pre:formacion}      & 197.00€ \\ \hline
      & \textbf{TOTAL}                                    &                           & \textbf{10 977.00€} \\
    \hline
    \caption{Resumen de presupuesto}
    \label{pre:cliente}
\end{longtable}

\vspace{-30pt}
El coste total del proyecto es de \textbf{10 977.00€}