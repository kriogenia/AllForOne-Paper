\chapter{Estado del sistema}
\label{ch:estado_sistema}

Al momento de la entrega de este documento y del sistema para la presentación del proyecto, \emph{AllForOne} no se encuentra en un estado de poder ser publicado o utilizado por el público general.

Si bien las funcionalidades clave planeadas para el sistema se encuentran desarrolladas y probadas, el sistema \textbf{sigue necesitando iteraciones y revisiones} que terminen de pulir estas y de agregar la seguridad o fiabilidad esperadas de un producto disponible públicamente. Por poner un ejemplo que fue nombrado en el \fref{ch:obstaculos}, las comunicaciones entre el servidor y la aplicación móvil no se encuentran a día de hoy cifradas y esta característica debería ser un requisito mínimo a alcanzar antes de dar el sistema por concluido.

De igual manera, una característica planeada que no se ha podido llevar a cabo por temas de tiempo y prioridades y que se plantea en el \fref{sec:avenencia_legal} es el desarrollo de todo el entramado necesario para garantizar los derechos y leyes relativos a los sistemas digitales o a los datos de los usuarios. No es siquiera planteable el lanzar una aplicación que registra datos de contacto de un paciente de Alzheimer sin cubrir las debidas protecciones de datos críticos como esos.

Otro aspecto necesario de cara a completar el sistema es el escalado del servidor a una instancia más preparada para un despliegue en producción. Actualmente el servidor se encuentra en una instancia de \textbf{tipo B}, dirigidas para pruebas; por lo que sería necesario plantear la mejora a una cuota de \textbf{tipo A}.

Por lo demás, y aunque algunas de las ampliaciones que se propondrán en \fref{ch:ampliaciones} deberían ser desarrolladas antes de alcanzar una versión \code{1.0} del sistema, el sistema se encuentra en un estado de completitud suficientemente satisfactorio para su uso por público limitado o grupos de control y, por tanto, suficiente de cara a esta entrega.