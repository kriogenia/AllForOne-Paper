\chapter{Conclusiones personales}
\label{ch:conclusiones_personales}

\textbf{Ha sido un largo camino}, y me gustaría poder hacer énfasis en el adjetivo de dicha afirmación. Ni siquiera ha sido mi primer intento en esa ingeniería, cuatro años antes de entrar a la Escuela ya me había aventurado en la facultad de informática de A Coruña en el mismo grado. El día que pisé por primera vez aquel lugar a casi trescientos kilómetros hace más de nueve años ya tenía claro cuál sería mi proyecto de fin de estudios. 

Aquí estoy por fin, redactando las conclusiones de este documento acerca del desarrollo de ese sistema para el apoyo a pacientes y familias afectadas por la terrible enfermedad de Alzheimer que siempre quise hacer. Lo que he creado dista mucho de parecerse mínimamente a aquella primera idea que tuve entonces. No se parece siquiera a ninguna de las iteraciones del concepto que pasaron por mi cabeza a lo largo de todos estos años. Hubo una época en la que incluso me planteaba la construcción de un dispositivo destinado a este fin.

Sin embargo, estoy muy contento de lo que ha salido aquí. Creo que es la mejor versión de todo lo que he pensado, y esto es porque se construye sobre la experiencia y sobre los conocimientos adquiridos en estos cuatro años de carrera. Es fácil soñar con características de tu aplicación ideal, pero cuando le aplicas la pintura de realidad y lo consigues plasmar en una función realizable con la aplicación de las buenas prácticas y estándares de calidad esperables, lo que resulta es incluso más atractivo. Porque es patente que todo este sistema se construye sobre los aprendizajes de este grado. 

En la elaboración de este proyecto trabajé sobre cosas en las que ya tenía experiencia, pero también he descubierto nuevas tecnologías y sistemas en los que mi conocimiento era nulo. \textbf{Socket.io} y los \textbf{WebSocket} son una tecnología con un muy buen funcionamiento y, sobre todo, con una curva de entrada muy halagüeña con la que se me hizo muy sencillo desarrollar las comunicaciones que necesitaba en muy poco tiempo. En similar manera descubrí el lenguaje \textbf{Kotlin} en profundidad, encontrándome con una herramienta de gran utilidad que aún con el proceso de aprendizaje redujo ampliamente mis tiempos de creación de código respecto a hacer lo mismo en Java.

Con otras muchas cosas ya tenía cierta experiencia, como con \textbf{Express}, \textbf{MongoDB} o \textbf{Android}. Para la base de cada uno de mis subsistemas decidí abogar por algún sistema que se me hubiese enseñado en la carrera y lo cierto es que fue la mejor decisión posible. En el proceso gané agilidad, pero no perdí la obtención de nuevos conocimientos, porque siempre hay más que aprender y con estas tecnologías también mejoré las bases que se me habían proporcionado en la carrera con buenas prácticas recabadas en diferentes documentaciones o artículos.

Aún con todo, y siguiendo la base de que la creación de este proyecto es sinónima con el aprendizaje y el crecimiento, donde más progresé como ingeniero informático con este desarrollo es en todos los pasos anteriores a ponerse a crear código. Cada diagrama que dibujaba en mi cuaderno de notas, cada requisito que me apuntaba en la primera lista que hice o cada clase que ya tenía perfectamente concebida antes de ponerme delante del \acrshort{ide} fue un gran progreso, ayuda y aprendizaje. Fue la demostración más palpable de que saldré de aquí como \textbf{ingeniero de software} y no únicamente como una persona capaz de hablar un idioma que entienda un sistema informático.

He disfrutado cada minuto de este desarrollo. Se me ha hecho una tarea titánica que me llegó a sobrepasar, pero lo he disfrutado. Me he frustrado, me he agotado, he aprendido y me he alegrado. No es la aplicación con la que soñaba cuando pisaba por primera vez la educación universitaria, pero sí consigue todos los objetivos que quería y entrega un producto mucho mejor de lo que podría haber esperado. Y todo esto lo he hecho yo, con mis conocimientos, averiguaciones y mis propias manos. Y no puedo estar más \textbf{orgulloso}.