\subsection{Aplicación móvil}

\subsubsection{Registro}

\begin{longtable}{|p{0.25\textwidth} p{0.75\textwidth}|}
    \hline
    \multicolumn{2}{|l|}{\textbf{Registro de un Paciente en la aplicación}} \\ \hline 
    Descripción                 & Realizar el proceso de registro de una cuenta nueva en la aplicación \\ \hline
    Caso de uso                 & \nameref{cu:registro} \\ \hline
    Entrada                     & Pulsar en \emph{Iniciar sesión} \\
                                & Selección de cuenta no registrada \\ 
                                & Introducir nombre \\
                                & Pulsar en \emph{Avanzar} \\
                                & Seleccionar rol Paciente \\
                                & Pulsar en \emph{Avanzar} \\
                                & Rellenar datos de contacto \\
                                & Pulsar en \emph{Avanzar} \\
                                & Aserción de datos \\
                                & Pulsar en \emph{Finalizar} \\
                                & Aserción de la tarjeta de Paciente \\ \hline
    Resultado esperado          & Debe completarse satisfactoriamente sin errores \\ \hline
    \caption{Prueba de integración de la aplicación: Registro de un Paciente en la aplicación}
    \label{cp:i:app:registro_paciente}
    
\end{longtable}

\newpage
\begin{longtable}{|p{0.25\textwidth} p{0.75\textwidth}|}
    \hline
    \multicolumn{2}{|l|}{\textbf{Registro de un Cuidador en la aplicación}} \\ \hline 
    Descripción                 & Realizar el proceso de registro de una cuenta nueva en la aplicación \\ \hline
    Caso de uso                 & \nameref{cu:registro} \\ \hline
    Entrada                     & Pulsar en \emph{Iniciar sesión} \\
                                & Selección de cuenta no registrada \\ 
                                & Introducir nombre \\
                                & Pulsar en \emph{Avanzar} \\
                                & Seleccionar rol Cuidador \\
                                & Pulsar en \emph{Avanzar} \\
                                & Rellenar datos de contacto \\
                                & Pulsar en \emph{Avanzar} \\
                                & Aserción de datos \\
                                & Pulsar en \emph{Finalizar} \\
                                & Aserción de presencia del botón \emph{Forjar vínculo} \\ \hline
    Resultado esperado          & Debe completarse satisfactoriamente sin errores \\ \hline
    \caption{Prueba de integración de la aplicación: Registro de un Cuidador en la aplicación}
    \label{cp:i:app:registro_cuidador}
\end{longtable}

\subsubsection{Inicio y cierre de sesión}

\begin{longtable}{|p{0.25\textwidth} p{0.75\textwidth}|}
    \hline
    \multicolumn{2}{|l|}{\textbf{Inicio de sesión en la aplicación}} \\ \hline 
    Descripción                 & Realizar el proceso de inicio de sesión en la aplicación \\ \hline
    Caso de uso                 & \nameref{cu:iniciar_sesion} \\ \hline
    Entrada                     & Pulsar en \emph{Iniciar sesión} \\
                                & Selección de cuenta de Paciente ya registrada \\ 
                                & Aserción de la tarjeta de Paciente \\ 
                                & Pulsar en \emph{Ajustes} \\ 
                                & Pulsar en \emph{Cerrar sesión} \\
                                & Aserción de la presencia del botón \emph{Iniciar sesión} \\ \hline
    Resultado esperado          & Debe completarse satisfactoriamente sin errores \\ \hline
    \caption{Prueba de integración de la aplicación: Inicio de sesión en la aplicación}
    \label{cp:i:app:inicio_sesion}
\end{longtable}

\newpage
\subsubsection{Vinculación}

\begin{longtable}{|p{0.25\textwidth} p{0.75\textwidth}|}
    \hline
    \multicolumn{2}{|l|}{\textbf{Mostrar código de vinculación en la aplicación}} \\ \hline 
    Descripción                 & Mostrar el código de vinculación de un Paciente en la aplicación \\ \hline
    Caso de uso                 & \nameref{cu:vincular_paciente} \\ \hline
    Entrada                     & Iniciar sesión con un Paciente no vinculado \\
                                & Navegar a \emph{Vínculos} \\ 
                                & Pulsar en \emph{Añadir vínculo} \\
                                & Aserción de presencia de código QR en pantalla \\ \hline
    Resultado esperado          & Debe completarse satisfactoriamente sin errores \\ \hline
    \caption{Prueba de integración de la aplicación: Mostrar código de vinculación en la aplicación}
    \label{cp:i:app:mostrar_codigo_vinculacion}
\end{longtable}

\vspace{-10pt}
\begin{longtable}{|p{0.25\textwidth} p{0.75\textwidth}|}
    \hline
    \multicolumn{2}{|l|}{\textbf{Desplegar escáner en la aplicación}} \\ \hline 
    Descripción                 & Abrir el escáner en la aplicación \\ \hline
    Caso de uso                 & \nameref{cu:vincular_cuidador} \\ \hline
    Entrada                     & Iniciar sesión con un Cuidador no vinculado \\
                                & Pulsar en \emph{Forjar vínculo} \\
                                & Pulsar en \emph{Conceder permiso} \\
                                & Aserción de apertura del escáner \\ \hline
    Resultado esperado          & Debe completarse satisfactoriamente sin errores \\ \hline
    \caption{Prueba de integración de la aplicación: Desplegar escáner en la aplicación}
    \label{cp:i:app:desplegar_escaner}
\end{longtable}

\vspace{-10pt}
\begin{longtable}{|p{0.25\textwidth} p{0.75\textwidth}|}
    \hline
    \multicolumn{2}{|l|}{\textbf{Eliminación de vínculo de un Paciente}} \\ \hline 
    Descripción                 & Eliminar un vínculo activo de un Paciente \\ \hline
    Caso de uso                 & \nameref{cu:desvincular} \\ \hline
    Entrada                     & Iniciar sesión con un Paciente vinculado \\
                                & Pulsar en \emph{Vínculos} \\
                                & Mantener pulsado en un vínculo \\
                                & Confirmar la eliminación \\
                                & Aserción de desaparición del vínculo \\
                                & Cambiar a cuenta vinculada \\
                                & Aserción de desaparición del vínculo \\ \hline
    Resultado esperado          & Debe completarse satisfactoriamente sin errores \\ \hline
    \caption{Prueba de integración de la aplicación: Eliminación de vínculo de un Paciente}
    \label{cp:i:app:eliminiacion_vinculo_paciente}
\end{longtable}

\vspace{-20pt}
\begin{longtable}{|p{0.25\textwidth} p{0.75\textwidth}|}
    \hline
    \multicolumn{2}{|l|}{\textbf{Eliminación de vínculo de un Cuidador}} \\ \hline 
    Descripción                 & Eliminar un vínculo activo de un Cuidador \\ \hline
    Caso de uso                 & \nameref{cu:desvincular} \\ \hline
    Entrada                     & Iniciar sesión con un Cuidador vinculado \\
                                & Pulsar en \emph{Ajustes} \\
                                & Pulsar en \emph{Eliminar vínculo} \\
                                & Confirmar la eliminación \\
                                & Aserción de desaparición del vínculo \\
                                & Cambiar a cuenta vinculada \\
                                & Pulsar en \emph{Vínculos} \\
                                & Aserción de desaparición del vínculo \\ \hline
    Resultado esperado          & Debe completarse satisfactoriamente sin errores \\ \hline
    \caption{Prueba de integración de la aplicación: Eliminación de vínculo de un Cuidador}
    \label{cp:i:app:eliminacion_vinculo_cuidador}
\end{longtable}

\subsubsection{Compartir ubicación}

\begin{longtable}{|p{0.25\textwidth} p{0.75\textwidth}|}
    \hline
    \multicolumn{2}{|l|}{\textbf{Compartir ubicación en la aplicación}} \\ \hline 
    Descripción                 & Compartir la ubicación de un usuario en la aplicación \\ \hline
    Caso de uso                 & \nameref{cu:compartir_ubicacion} \\ \hline
    Entrada                     & Iniciar sesión con un usuario vinculado \\
                                & Navegar a \emph{Localización} \\ 
                                & Pulsar en \emph{Conceder permiso} \\
                                & Aserción de la localización \\
                                & Desplazar GPS \\
                                & Aserción de la localización \\ \hline
    Resultado esperado          & Debe completarse satisfactoriamente sin errores \\ \hline
    \caption{Prueba de integración de la aplicación: Compartir ubicación en la aplicación}
    \label{cp:i:app:compartir_ubicacion}
\end{longtable}

\newpage
\subsubsection{Gestionar tareas}

\begin{longtable}{|p{0.25\textwidth} p{0.75\textwidth}|}
    \hline
    \multicolumn{2}{|l|}{\textbf{Gestionar las tareas desde Tareas}} \\ \hline 
    Descripción                 & Probar todas las opciones de gestión de tareas en la aplicación \\ \hline
    Caso de uso                 & \nameref{cu:listar_tareas}, \nameref{cu:crear_tarea}, \nameref{cu:marcar_tarea}, \nameref{cu:desmarcar_tarea}, \nameref{cu:eliminar_tarea} \\ \hline
    Entrada                     & Iniciar sesión con un usuario vinculado \\
                                & Navegar a \emph{Tareas} \\ 
                                & Pulsar en \emph{Crear tarea} \\
                                & Introducir título y descripción \\
                                & Pulsar en \emph{Confirmar} \\
                                & Aserción de la presencia de la tarea \\
                                & Cambiar a cuenta vinculada \\
                                & Navegar a \emph{Tareas} \\ 
                                & Aserción de la presencia de la tarea \\
                                & Marcar tarea como hecha \\
                                & Aserción del estado de la tarea \\
                                & Cambiar a cuenta inicial \\
                                & Navegar a \emph{Tareas} \\ 
                                & Aserción del estado de la tarea \\
                                & Marcar tarea como no hecha \\
                                & Aserción del estado de la tarea \\
                                & Pulsar en el botón de eliminar tarea \\
                                & Confirmar eliminación \\
                                & Aserción de la ausencia de la tarea \\
                                & Cambiar a cuenta inicial \\
                                & Navegar a \emph{Tareas} \\ 
                                & Aserción de la ausencia de la tarea \\
                                \hline
    Resultado esperado          & Debe completarse satisfactoriamente sin errores \\ \hline
    \caption{Prueba de integración de la aplicación: Gestionar las tareas desde Tareas}
    \label{cp:i:app:gestionar_tareas_tasks}
\end{longtable}

\newpage
\begin{longtable}{|p{0.25\textwidth} p{0.75\textwidth}|}
    \hline
    \multicolumn{2}{|l|}{\textbf{Gestionar las tareas desde el Feed}} \\ \hline 
    Descripción                 & Probar todas las opciones de gestión de tareas desde el Feed de la aplicación \\ \hline
    Caso de uso                 & \nameref{cu:listar_tareas}, \nameref{cu:crear_tarea}, \nameref{cu:marcar_tarea}, \nameref{cu:desmarcar_tarea}, \nameref{cu:eliminar_tarea} \\ \hline
    Entrada                     & Iniciar sesión con un usuario vinculado \\
                                & Navegar a \emph{Feed} \\ 
                                & Pulsar en \emph{Modo tarea} \\
                                & Introducir título y descripción \\
                                & Pulsar en \emph{Enviar} \\
                                & Aserción del listado de la tarea \\
                                & Cambiar a cuenta vinculada \\
                                & Navegar a \emph{Feed} \\ 
                                & Aserción de la presencia de la tarea \\
                                & Marcar tarea como hecha \\
                                & Aserción del estado de la tarea \\
                                & Cambiar a cuenta inicial \\
                                & Navegar a \emph{Feed} \\ 
                                & Aserción del estado de la tarea \\
                                & Marcar tarea como no hecha \\
                                & Aserción del estado de la tarea \\
                                & Mantener presionada la tarea \\
                                & Confirmar eliminación \\
                                & Aserción de la ausencia de la tarea \\
                                & Cambiar a cuenta inicial \\
                                & Navegar a \emph{Feed} \\ 
                                & Aserción de la ausencia de la tarea \\
                                \hline
    Resultado esperado          & Debe completarse satisfactoriamente sin errores \\ \hline
    \caption{Prueba de integración de la aplicación: Gestionar las tareas desde el Feed}
    \label{cp:i:app:gestionar_tareas_feed}
\end{longtable}

\newpage
\subsubsection{Envío y recepción de mensaje}

\begin{longtable}{|p{0.25\textwidth} p{0.75\textwidth}|}
    \hline
    \multicolumn{2}{|l|}{\textbf{Enviar y recibir mensajes en la aplicación}} \\ \hline 
    Descripción                 & Probar el sistema de mensajería de la aplicación \\ \hline
    Caso de uso                 & \nameref{cu:enviar_mensajes} \\ \hline
    Entrada                     & Iniciar sesión con un usuario vinculado \\
                                & Navegar a \emph{Feed} \\ 
                                & Introducir un mensaje \\
                                & Pulsar en \emph{Enviar} \\
                                & Aserción del listado del mensaje \\
                                & Cambiar a cuenta vinculada \\
                                & Navegar a \emph{Feed} \\ 
                                & Aserción de la del mensaje \\
                                & Introducir un mensaje \\
                                & Pulsar en \emph{Enviar} \\
                                & Aserción del listado del mensaje \\
                                & Cambiar a cuenta vinculada \\
                                & Navegar a \emph{Feed} \\ 
                                & Aserción de la del mensaje \\
                                \hline
    Resultado esperado          & Debe completarse satisfactoriamente sin errores \\ \hline
    \caption{Prueba de integración de la aplicación: Enviar y recibir mensajes en la aplicación}
    \label{cp:i:app:enviar_recibir_mensajes}
\end{longtable}

\newpage
\subsubsection{Gestionar notificaciones}

\begin{longtable}{|p{0.25\textwidth} p{0.75\textwidth}|}
    \hline
    \multicolumn{2}{|l|}{\textbf{Gestionar notificaciones en la aplicación}} \\ \hline 
    Descripción                 & Probar las notificaciones de la aplicación \\ \hline
    Caso de uso                 & \nameref{cu:consultar_notificaciones} \\ \hline
    Entrada                     & Iniciar sesión con un usuario vinculado \\
                                & Crear tres tareas \\ 
                                & Cambiar a cuenta vinculada \\
                                & Aserción del número de notificaciones \\
                                & Abrir \emph{Notificaciones} \\ 
                                & Aserción de las notificaciones \\
                                & Marcar una como leída \\
                                & Aserción de la ausencia de la notificación \\
                                & Cerrar \emph{Notificaciones} \\
                                & Aserción del número de notificaciones \\
                                & Abrir \emph{Notificaciones} \\ 
                                & Marcar todas como leídas \\
                                & Aserción de la ausencia de las notificaciones \\
                                & Cerrar \emph{Notificaciones} \\
                                & Aserción del número de notificaciones \\
                                \hline
    Resultado esperado          & Debe completarse satisfactoriamente sin errores \\ \hline
    \caption{Prueba de integración de la aplicación: Gestionar notificaciones en la aplicación}
    \label{cp:i:app:gestionar_notificaciones}
\end{longtable}

\vspace{-15pt}
\subsubsection{Consultar vínculos}

\begin{longtable}{|p{0.25\textwidth} p{0.75\textwidth}|}
    \hline
    \multicolumn{2}{|l|}{\textbf{Consultar los vínculos en la aplicación}} \\ \hline 
    Descripción                 & Consultar los vínculos en la aplicación \\ \hline
    Caso de uso                 & \nameref{cu:consultar_cuidador} \\ \hline
    Entrada                     & Iniciar sesión con un Paciente con dos vínculos \\
                                & Abrir \emph{Vínculos} \\ 
                                & Aserción de los vínculos (dos) \\
                                & Cambiar a cuenta vinculada \\
                                & Abrir \emph{Vínculos} \\ 
                                & Aserción de los vínculos (uno) \\
                                \hline
    Resultado esperado          & Debe completarse satisfactoriamente sin errores \\ \hline
    \caption{Prueba de integración de la aplicación: Consultar los vínculos en la aplicación}
    \label{cp:i:app:consultar_vinculos}
\end{longtable}

\subsubsection{Consultar paciente}

\begin{longtable}{|p{0.25\textwidth} p{0.75\textwidth}|}
    \hline
    \multicolumn{2}{|l|}{\textbf{Consultar paciente en la aplicación}} \\ \hline 
    Descripción                 & Realizar el proceso de inicio de sesión en la aplicación \\ \hline
    Caso de uso                 & \nameref{cu:consultar_paciente} \\ \hline
    Entrada                     & Iniciar sesión con Cuidador vinculado \\
                                & Aserción de la tarjeta de Paciente \\ 
    Resultado esperado          & Debe completarse satisfactoriamente sin errores \\ \hline
    \caption{Prueba de integración de la aplicación: Consultar paciente en la aplicación}
    \label{cp:i:app:consultar_paciente}
\end{longtable}