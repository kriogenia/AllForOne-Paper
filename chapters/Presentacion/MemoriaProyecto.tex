\chapter{Memoria del proyecto}
\label{chap:memoria}

\section{Resumen de la motivación, objetivos y alcance del proyecto}

\textbf{La enfermedad de Alzheimer es un trastorno neurodegenerativo} que reduce grave y progresivamente las capacidades cognitivas de sus pacientes. Su duración media tras el diagnóstico es de diez años, a lo largo de los cuales se va produciendo un deterioro en la memoria, la independencia, los trastornos conductuales y las capacidades físicas del afectado.

El diagnóstico de esta enfermedad es también el inicio de un cambio repentino en el entorno del diagnosticado. Actualmente no existe ningún tratamiento que detenga la enfermedad, aunque algunos fármacos contribuyan a retrasar el avance de la misma. Aún así, la mayor parte de los paliativos de la misma son de tipo \textbf{conductual} e involucran no solo al paciente sino también a \textbf{su entorno cercano}.

El objetivo de este proyecto es el \textbf{desarrollo de un sistema software} que pueda facilitar algunos de los aspectos en los que familiares, amigos y demás allegados del paciente son clave para servirle de apoyo en las etapas tempranas de la enfermedad.

El sistema a desarrollar será una \textbf{aplicación móvil} con funciones de contacto, gestión de tareas, geolocalización y mensajería. Todas estas funciones de la aplicación estarán respaldadas por \textbf{un servidor} desplegado en la nube que se encargará de la lógica subyacente y de la persistencia de los datos y la comunicación de estos entre usuarios. Dicho servidor ofrecerá dos interfaces de comunicación: una basada en WebSockets para los envíos de comunicaciones más inmediatas de los usuarios con otros usuarios y con el sistema; y otra \acrshort{http} con una \acrshort{api} \acrshort{rest} para el resto de operaciones.

Para poder crear el sistema deseado será necesario utilizar \textbf{diversas tecnologías} como los anteriormente mencionados WebSockets, los sistemas de mapas y geolocalización, las corrutinas, las bases de datos NoSQL o los servicios en la nube. La mayor parte de esta son nuevas para el equipo de desarrollo y de esta forma representan un obstáculo a la vez que una oportunidad de aprendizaje.

La creación del sistema se basará en la planificación, análisis y diseño presentados en este documento. Documento que junto al sistema descrito en las líneas previa conforman este proyecto de desarrollo de una \textbf{aplicación asimétrica móvil de asistencia para familias con personas afectadas de Alzheimer}.

\section{Estructura del documento}

El documento que nos ocupa consta de los siguientes secciones y capítulos:

\begin{enumerate}
    \item \textbf{Presentación}
    \begin{enumerate}
    	\item \textbf{\nameref{chap:memoria}}. Breve introducción al documento.
    	\item \textbf{\nameref{ch:introduccion}}. Presentación general del proyecto que se desarrollará en este documento.
    	\item \textbf{\nameref{ch:aspectos_teoricos}}. Explicación de conceptos de la especialidad importantes para la comprensión del proyecto.
	\end{enumerate}
    \item \textbf{Planificación}
    \begin{enumerate}
    	\item \textbf{\nameref{ch:requisitos_iniciales}}. Lista inicial de requisitos del sistema.
    	\item \textbf{\nameref{ch:evaluacion_alternativas}}. Análisis de opciones tecnológicas para el desarrollo del proyecto.
    	\item \textbf{\nameref{ch:analisis_riesgos}}. Perspectiva de riesgos del desarrollo y plan de actuación.
    	\item \textbf{\nameref{ch:solucion_propuesta}}. Resumen del sistema a desarrollar.
    	\item \textbf{\nameref{ch:planificacion_temporal}}. Propuesta de calendario para la realización del proyecto.
    	\item \textbf{\nameref{ch:resumen_presupuesto}}. Puntos clave del presupuesto estimado.
	\end{enumerate}
    \item \textbf{Análisis}
    \begin{enumerate}
    	\item \textbf{\nameref{ch:requisitos}}. Listado definitivo de requisitos planeados para el sistema.
    	\item \textbf{\nameref{ch:subsistemas}}. Definición de los diferentes subsistemas que conformarán \emph{AllForOne}.
    	\item \textbf{\nameref{ch:analisis_interfaz_usuario}}. Propuesta de pantallas y de navegación para la aplicación.
    	\item \textbf{\nameref{ch:analisis_diagrama_clases}}. Diseño previo de las clases que se estimará que tenga el sistema.
    	\item \textbf{\nameref{ch:especificacion_plan_pruebas}}. Plan de desarrollo y realización de pruebas del sistema.
    	\item \textbf{\nameref{ch:especificacion_plan_despliegue}}. Plan de despliegue del servidor.
	\end{enumerate}
    \item \textbf{Diseño}
    \begin{enumerate}
    	\item \textbf{\nameref{ch:arquitectura}}. Propuesta de arquitectura para el sistema.
    	\item \textbf{\nameref{ch:diseño_clases}}. Diseño completo del sistema de clases y componentes.
    	\item \textbf{\nameref{ch:modelo_datos}}. Definición de las entidades manejadas por las bases de datos.
    	\item \textbf{\nameref{ch:interfaces_comunicacion}}. Definición de las \acrshort{api} de comunicación entre aplicación y servidor.
    	\item \textbf{\nameref{ch:interaccion_estados}}. Explicación de las interacciones, procedimientos y estados relacionados con los casos de uso a implementar.
    	\item \textbf{\nameref{ch:diseño_interfaz_usuario}}. Diseño definitivo de las pantallas de la aplicación.
    	\item \textbf{\nameref{ch:especificacion_tecnica_plan_pruebas}}. Recolección de los casos de prueba anticipados para el desarrollo.
	\end{enumerate}
    \item \textbf{Implementación}
    \begin{enumerate}
    	\item \textbf{\nameref{ch:tecnologias}}. Enumeración y descripción de las tecnologías usadas en el desarrollo.
    	\item \textbf{\nameref{ch:herramientas_desarrollo}}. Enumeración y descripción de las herramientas usadas para el desarrollo.
    	\item \textbf{\nameref{ch:obstaculos}}. Descripción de los obstáculos encontrados a lo largo del desarrollo.
    	\item \textbf{\nameref{ch:desarrollo_pruebas}}. Resultados y conclusiones del apartado de pruebas del proyecto.
	\end{enumerate}
    \item \textbf{Manuales}
    \begin{enumerate}
    	\item \textbf{\nameref{ch:manual_instalacion}}. Guía de instalación de la aplicación.
    	\item \textbf{\nameref{ch:manual_usuario}}. Guía de uso de la aplicación.
    	\item \textbf{\nameref{ch:manual_despliegue}}. Guía de procedimiento para el despliegue del servidor.
    	\item \textbf{\nameref{ch:manual_desarrollador}}. Conjunto de guías de ayuda para la continuación del desarrollo.
	\end{enumerate}
    \item \textbf{Conclusión}
    \begin{enumerate}
    	\item \textbf{\nameref{ch:estado_sistema}}. Descripción del estado del sistema en el momento de la entrega.
    	\item \textbf{\nameref{ch:conclusiones_personales}}. Experiencia personal del desarrollo y el proceso que condujo al mismo.
    	\item \textbf{\nameref{ch:ampliaciones}}. Lista de posibles ampliaciones planeadas o propuestas para el sistema.
	\end{enumerate}
    \item \textbf{Anexos}
    \begin{enumerate}
    	\item \textbf{\nameref{ch:anexo_presupuesto}}. Versión detallada del presupuesto.
    	\item \textbf{\nameref{ch:licencia}}. Licencia de uso del sistema desarrollado.
    	\item \textbf{\nameref{ch:modelado_datos}}. Esquemas de las entidades modeladas.
    	\item \textbf{\nameref{ch:accion_despliegue}}. Fichero de despliegue del servidor.
    	\item \textbf{\nameref{ch:documentacion_adicional}}. Listado de documentación adicional entregada junto a este documento.
	\end{enumerate}
\end{enumerate}