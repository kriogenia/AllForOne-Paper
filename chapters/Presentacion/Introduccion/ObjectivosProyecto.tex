\section{Objetivos del proyecto}

El objetivo del proyecto es completar el desarrollo de un prototipo de aplicación que ofrezca a sus usuarios, pacientes y cuidadores, herramientas para facilitar en la medida de lo posible los siguientes síntomas de las etapas temprana y media de la enfermedad:

\begin{enumerate}
    \item \textbf{Olvido de datos personales}. El paciente podrá guardar sus datos elementales como la dirección o el número de teléfono para tenerlos siempre a mano en caso de necesidad. Los datos de sus cuidadores también se le ofrecerán para que también puedan ser de ayuda en dichos momentos.
    \item \textbf{Desorientación}. En caso de que el paciente se pierda, se podrá utilizar un servicio para compartir la ubicación a través del que sus cuidadores podrán localizarlo.
    \item \textbf{Olvido de tareas y responsabilidades}. Pacientes y cuidadores tendrán a su disponibilidad una lista de tareas compartida en la que todos podrán crear y gestionar tareas, favoreciendo que no sean olvidadas y puedan ser llevadas a cabo.
\end{enumerate}

Todo esto se desarrollará con un enfoque centrado en potenciar y facilitar la cooperación de todos los involucrados con un paciente. Esto implica añadir una vía de comunicación compartida y un canal de notificaciones para favorecer que todos los cuidadores estén siempre al tanto de las novedades del paciente.

Por último, un objetivo importante del proyecto es ofrecer un producto que sea de utilidad para todas las personas que deban interaccionar con esta enfermedad, no solo directa, sino también indirectamente. Esto quiere decir que otro de los objetivos de este proyecto es que la aplicación pueda ser un punto de valor para el entorno tecnológico enfocado en el Alzheimer. 

\textbf{No existe un interés comercial en el desarrollo de este sistema} y el enfoque económico y presupuestario del mismo siempre irán enfocado en el afán survivabilista y de coexistencia con el resto de aplicaciones de índole similar, por ello, en el resto del documento nunca se hablará de mercado, sino de ámbito. Para sumar aún más sobre este objetivo, \emph{All for One} tiene como objetivo ser un proyecto de \textbf{código abierto} y así lo ha sido desde el día uno.