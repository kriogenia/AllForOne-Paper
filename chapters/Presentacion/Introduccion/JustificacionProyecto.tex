\section{Justificación del proyecto}
\label{sec:justification}

En el tiempo que se tarda en leer esta primera oración completa un nuevo caso de Alzheimer ha sido diagnosticado. \textbf{Siete segundos}. Si extendiésemos los casos de Alzheimer anuales del mundo equitativamente en el tiempo que dura dicho año tendríamos un nuevo diagnóstico cada siete segundos.

Según el estudio \emph{El cuidador en España. Contexto actual y perspectivas de futuro. Propuestas de intervención.}\cite{ceafa2016cuidador} de \acrshort{ceafa} en 2016, en España existen más de 1,2 millones de personas afectadas por Alzheimer, lo que equivale a que uno de cada cuatro hogares tenga relación directa con un paciente de esta enfermedad.

\textbf{La motivación de este proyecto parte de esta situación}. La aparición de esta enfermedad neurodegenerativa implica un cambio radical en todos esos miles de hogares que son tocados por ella. Las familias y el entorno cercano deben adaptarse para poder proporcionar toda la ayuda necesaria para el paciente. La comunicación es la piedra básica sobre la que es necesario construir los cimientos que permitan capear lo mejor posible este inesperado temporal. Descubrir esta demencia implica la necesidad de crear una coordinación, no solo con el paciente, sino con el resto de cuidadores.

Como se comentó en el \autoref{chap:memoria}, el mal de Alzheimer tiene un diagnóstico terminal y las principales indicaciones para combatir la enfermedad pasan por actividades, conductas o actitudes que paciente y allegados deben llevar a cabo. Es habitual que se inste a formar parte de las diversas y numerosas asociaciones dedicadas a esta enfermedad que se encuentran repartidas por todo el territorio español. La lucha contra el Alzheimer de las familias afectadas a día de hoy no es pasiva ni individual, \textbf{es un esfuerzo colectivo activo} de todo el entorno para paliar los estragos que deja tras de sí el trastorno.

En las etapa más temprana de la enfermedad la persona afectada aún se puede \textbf{desenvolver de forma independiente}. Es habitual que una persona recién diagnosticada continúe trabajando o relacionándose de forma muy próxima a lo que era habitual en su vida por un tiempo. Sin embargo, experimentará una serie de dificultades que se irán haciendo más patentes y habituales con el paso del tiempo.

\newpage

La incapacidad de encontrar una palabra o nombre correcto. Olvidar inmediatamente conocimientos recién adquiridos y extraviar objetos. Ser incapaces de planificar y, posteriormente, de llevar a cabo lo planeado. Sufrir cambios de humor o pérdidas de ánimo. Son algunas de las situaciones que los afectados por esta enfermedad vivirán \textbf{cada vez más a menudo} según pase el tiempo.

Posteriormente llega la etapa media, \textbf{la más prolongada en el tiempo}, a lo largo de la cual el enfermo va perdiendo su autonomía y se vuelve muy necesario otorgarle máxima atención. En este periodo el enfermo comienza a sufrir síntomas más inhabilitantes como son: desorientarse y perderse incluso en entornos conocidos, olvidar datos personales propios, no reconocer su propio rostro, desórdenes del sueño, desubicación temporal, pérdida de capacidad de uso de herramientas, problemas para controlar las necesidades básicas y la higiene o la incapacidad total de realizar cosas planificadas como tomar su medicación.

Está en la mano de sus personas cercanas el ayudarles a paliar y minimizar las dificultades que estas situaciones pueden generar. A medida que la demencia avanza el afectado es cada vez menos capaz de afrontar dichas dificultades y más vital se vuelve que su entorno le ayude a llevar a cabo todas esas acciones que la enfermedad le impide llevar a cabo con normalidad.

Facilitar todo esto es el objetivo principal de \textbf{All for One}, este proyecto. El nombre proviene del lema que \emph{Los Tres Mosqueteros} de Alejandro Dumas proclamaban con sus roperas cruzadas: 

\begin{quotation} 
    \centering
    \emph{Unus pro omnibus, omnes pro uno}
    \vspace{-20pt}
    \begin{flushright}
        - Alexandre Dumas, \textit{1844}
    \end{flushright} 
\end{quotation} 

Uno para todos, todos para uno. La segunda mitad del lema evoca precisamente el ánimo que este sistema persigue. La colaboración y dedicación por y para alguien apreciado. Un esfuerzo de todos los seres queridos por esa persona víctima de la peor de las suertes.

La propuesta de proyecto es una \textbf{aplicación móvil} que ofrezca herramientas para facilitar, a los aquejados de Alzheimer y a su entorno, la larga y dura travesía a lo largo de esas etapas tempranas de la enfermedad. \emph{All for One} busca ser la ropera que permita a todos blandirse en duelo contra el mal que persigue a uno.