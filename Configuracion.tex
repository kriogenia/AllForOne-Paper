% Color definition
\definecolor{gray}{RGB}{220,220,220}
\definecolor{ThemeColor1}{cmyk}{0.98, 0.63, 0.71, 0.28}
\definecolor{ThemeColor2}{cmyk}{0, 0.25, 0.4, 0}

% Subsection enumeration
\setcounter{secnumdepth}{3} 

% Code command
\newcommand{\code}[1]{\texttt{#1}}

% Translate references
\newcommand{\fref}[1]{\foreignlanguage{spanish}{\Cref{#1}}}

% Extra TODOs
\newcommand{\todoopt}[2][]
{\todo[color=yellow, inline, #1]{#2}}

\newcommand{\todoreq}[2][]
{\todo[inline, #1]{#2}}

% Director
\newcommand{\director}[3]{
	\csdef{directorName}{#1}	
	\csdef{directorTitle}{#2}
	\csdef{directorField}{#3}
}

\newcommand{\printdirector}{
	\csuse{directorTitle}
	\csuse{directorName}
	\csuse{directorCharge}
}

% Author
\renewcommand{\author}[2]{
\def\dissertationAuthor{#1}
\def\dissertationAuthorEducation{#2}
}

% Title
\newcommand{\setTitle}[1]{
\def\dissertationTitle{#1}
}

% Subtitle
\newcommand{\setSubtitle}[1]{
\def\dissertationSubtitle{#1}
}

% University
\newcommand{\setUniversity}[1]{
\def\university{#1}
}

% Indexes name change
\addto\captionsspanish{
  \renewcommand{\contentsname}{Índice de contenido}
  \renewcommand{\listfigurename}{Índice de figuras}
  \renewcommand{\listtablename}{Índice de tablas}
}

% Section name formatting:
\titleformat{\chapter}[block]
  {\normalfont\Huge\bfseries\singlespacing}{\thechapter.}{1em}{\Huge}
\titlespacing*{\chapter}{0pt}{-62pt}{0pt}

\titleformat{\section}[block]
  {\normalfont\large\bfseries\color{ThemeColor1}}{\thesection.}{4pt}{\large}
\titlespacing*{\section}{0pt}{\baselineskip}{0pt}

\titleformat{\subsection}[block]
  {\normalfont\normalsize\bfseries\color{ThemeColor1}}{\thesubsection.}{4pt}{\normalsize}
\titlespacing*{\subsection}{0pt}{0pt}{0pt}

\titleformat{\subsubsection}[block]
  {\normalfont\normalsize\bfseries}{\thesubsubsection.}{4pt}{\normalsize}
\titlespacing*{\subsubsection}{0pt}{0pt}{0pt}

\def\tablename{Tabla}

% Header and footer formatting
\renewcommand\chaptermark[1]{\markboth{\thechapter\ #1}{}}

\fancyhf{}
\fancyhead[LE]{
    \includegraphics[height=16mm]{style/eii.png}
}
\fancyhead[LO]{
    \includegraphics[height=16mm]{images/style/logo_uniovi.jpg}
}
\fancyhead[C]{\Longstack[c]{
    \emph{Escuela de Ingeniería Informática} \newline
    \textsc{Universidad de Oviedo}}}
\fancyhead[R]{\bfseries{Pg. \thepage \hspace{1pt} de \pageref{LastPage}}}

\fancyfoot[L]{\slshape\nouppercase{\leftmark}}
  \fancyfoot[R]{Ricardo Soto Estévez}

\renewcommand{\headrulewidth}{0pt}
\renewcommand{\footrulewidth}{0.4pt}

% Overriding of fancy to add the headers and footers
\fancypagestyle{plain}{%
  \fancyhf{}
  \fancyhead[LE]{
    \includegraphics[height=16mm]{style/eii.png}
  }
  \fancyhead[LO]{\includegraphics[height=16mm]{images/style/logo_uniovi.jpg}}
  \fancyhead[C]{\Longstack[c]{
    \emph{Escuela de Ingeniería Informática} \newline
    \textsc{Universidad de Oviedo}}}
  \fancyhead[R]{\bfseries{Pg. \thepage \hspace{1pt} de \pageref{LastPage}}}
  
  \fancyfoot[L]{\slshape\nouppercase{\leftmark}}
  \fancyfoot[R]{Ricardo Soto Estévez}
  
  \renewcommand{\headrulewidth}{0pt}% default is 0pt
  \renewcommand{\footrulewidth}{0.4pt}% default is 0pt
}

% Href style
\hypersetup{
    final,
    colorlinks,
    citecolor= ThemeColor1,
    filecolor= ThemeColor1,
    linkcolor= black,
    urlcolor= ThemeColor1
}

% Lists style
\setlist[enumerate]{topsep=0pt}
\setlist[itemize]{topsep=0pt}

% Tables style
\setlength{\LTcapwidth}{\textwidth}

% TOCs style
\usepackage{tocloft}
\advance\cftsecnumwidth 0.5em\relax
\advance\cftsubsecindent 0.5em\relax
\advance\cftsubsecnumwidth 0.5em\relax
\advance\cftfignumwidth 0.5em\relax
\advance\cfttabnumwidth 0.7em\relax

% Style declaration
\pagestyle{fancy}
\restylefloat{table}

%%%%%%%%%%%%
% Listings %
%%%%%%%%%%%%

\renewcommand\lstlistingname{Fragmento}
\renewcommand\lstlistlistingname{Índice de fragmentos}

% Javascript code block
\definecolor{lightgray}{rgb}{.9,.9,.9}
\definecolor{darkgray}{rgb}{.4,.4,.4}
\definecolor{purple}{rgb}{0.65, 0.12, 0.82}
\lstdefinelanguage{TypeScript}{
  keywords={break, case, catch, continue, debugger, default, delete, do, else, false, finally, for, function, if, in, instanceof, new, null, return, switch, this, throw, true, try, typeof, var, void, while, with, @modelOptions, @prop},
  morecomment=[l]{//},
  morecomment=[s]{/*}{*/},
  morestring=[b]',
  morestring=[b]",
  ndkeywords={class, export, boolean, throw, implements, import, this, enum, interface, string, number},
  keywordstyle=\color{blue}\bfseries,
  ndkeywordstyle=\color{darkgray}\bfseries,
  identifierstyle=\color{black},
  commentstyle=\color{purple}\ttfamily,
  stringstyle=\color{red}\ttfamily,
  sensitive=true
}

\lstset{
   language=TypeScript,
   %backgroundcolor=\color{lightgray},
   extendedchars=true,
   basicstyle=\footnotesize\ttfamily,
   showstringspaces=false,
   showspaces=false,
   numbers=left,
   numberstyle=\footnotesize,
   numbersep=9pt,
   tabsize=2,
   breaklines=true,
   showtabs=false,
   captionpos=b
}

\lstdefinelanguage{Yaml}{
  keywords={on, push, branches, workflow_dispatch, jobs, build, runs, steps, deploy, needs, environment },
  morecomment=[l]{\#},
  morestring=[b]',
  morestring=[b]",
  ndkeywords={uses, name, with, run, url, path, node, version, app, slot, publish-profile, package },
  keywordstyle=\color{blue}\bfseries,
  ndkeywordstyle=\color{darkgray}\bfseries,
  identifierstyle=\color{black},
  commentstyle=\color{purple}\ttfamily,
  stringstyle=\color{red}\ttfamily,
  sensitive=true
}

\lstset{
   language=Yaml,
   %backgroundcolor=\color{lightgray},
   extendedchars=true,
   basicstyle=\footnotesize\ttfamily,
   showstringspaces=false,
   showspaces=false,
   numbers=left,
   numberstyle=\footnotesize,
   numbersep=9pt,
   tabsize=2,
   breaklines=true,
   showtabs=false,
   captionpos=b
}
